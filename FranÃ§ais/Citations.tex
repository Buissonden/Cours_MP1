\documentclass[a4paper,12pt]{book}
\usepackage{ae}
\usepackage{aeguill}
\usepackage{amsthm}
\usepackage[utf8]{inputenc}
\usepackage[T1]{fontenc}
\usepackage[francais]{babel}
\usepackage[utf8]{inputenc}
\usepackage{graphicx}
\usepackage{hyperref}
\usepackage{xcolor}
\usepackage[left=1cm, right= 1cm, top=2cm, bottom = 2cm]{geometry}
\usepackage{array,multirow}
\usepackage{amsmath,amsthm,amssymb}
\usepackage{amsfonts}
\usepackage{stmaryrd}
\usepackage{tcolorbox}
\usepackage{lmodern}



\newcommand{\Cit}[2]{\begin{tcolorbox}[sharp corners, colback=white,colframe=red!90!black!75, title=Citations : #1]#2\end{tcolorbox}}
\newcommand{\Ana}[2]{\begin{tcolorbox}[sharp corners, colback=white,colframe=red!90!black!75, title=Analyse : #1]#2\end{tcolorbox}}


\renewcommand{\thechapter}{\Roman{chapter}}
\renewcommand{\thesection}{\Roman{section}}
\renewcommand{\thesubsection}{\Roman{section}.\arabic{subsection}}
\renewcommand{\thesubsubsection}{\Roman{section}.\arabic{subsection}.\Alph{subsubsection}}

\begin{document}
\Cit{La Condition Ouvrière 1}{\begin{enumerate}
\item "Cette expérience, qui correspond par bien des côtés à ce que j'en attendais, en diffère quand même par un abîme : c'est la réalité et non plus l'imagination." p52
\item "on dégrade l'inexprimable à vouloir l'exprimer." p52
\item "C'est seulement le samedi après-midi et le dimanche que je respire, me retrouve moi-même, réacquirs la faculté de rouler dans mon esprit des morceaux d'idées." p53
\item "Je ne sais que trop [...] ce que c'est que de savourer ainsi la mort tout vivant, de voir des années s'étendre devannt soi, d'avoir mille fois de quoi les remplir, et de penser que la faiblesse physique forcera à les laisser vides, que les franchir simplement jour par jour sera une tâche écrasante." p56
\item "un endroit où on se heurte durement, douloureusement, mais quand même joyeusement à la vraie vie." p57
\item "Garde cette lettre - je te la redemanderai peut-être si, un jour, je veux rassembler tous mes souvenirs de cette vie d'ouvrière. Par pour publier quelque chose là-dessus (du moins je ne pense pas), mais pour me défendre moi-même de l'oubli." p61
\item "Non, le tragique de cette situation, c'est que le travail est trop machinal pour offrir matière à la pensée, et que néanmoins il interdit toute autre pensée." p67
\item "Penser, c'est allez moins vite ; or il y a des normes de vitesse, établies par des bureaucrates impitoyables" p67
\item "La bonté surtout, dans une usine, est quelque chose de réel quand elle existe ; car le moindre acte de bienveillance, depuis un simple sourire jusqu'à un service rendu, exige qu'on triomphe de la fatigue, de l'obsession du salaire, de tout ce qui accable et incite à se replier sur soi." p68
\item "Car la réalité de la vie, ce n'est pas la sensation, mais l'activité - j'entends l'activité et dans la pensée et dans l'action. Ceux qui vivent de sensations ne sont, matériellement et moralement, que des parasites par rapport aux hommes travailleurs et créateurs qui seuls sont des hommes." p69
\item "Vous comprenez, on nous fait une grâce en nous permettant de nous crever, et il faut dire merci." p74 
\item "Quand on a sa vie à gagner, il faut ce qu'il faut" p84
\item "Sentiment d'avoir été un être libre 24H (le dimanche), et de devoir me réadapter à une condition servile." p89
\item "Bon coulé, mais par faute du chrono" p92
\item "Incident bureaucratique : 10 rondelles manquantes" p97
\item "L'épuisement finit par me faire oublier les raisons véritables de mon séjour en usine, rend presque invincible pour moi la tentation la plus forte que comporte cette vie : celle de ne plus penser, seul et unique moyen de ne pas en souffrir." p103
\item "compte tenu du fait que j'ai perdu 10 mn à toucher ma payer le matin à 11H" p105
\item "J'arrive enfin à aller un peu vite (à la fin, je fais plus que 3 F l'heure), mais l'amertume au coeur." p123
\item "Une oppression évidemment inexorable et invincible n'engendre pas comme réaction immédiate la révolte, mais la soumission." p132
\item "On a tué le cinéma en le rendant parlant, au lieu de le laisser ce qu'il est véritablement, la plus belle application de la photographie." p132
\item "Puis, sérieusement : ça fait 5 ans que je n'ai pas dansé. On a envie de danser, et puis on danse devant la lessive." p138
\end{enumerate}}
\Cit{La Condition Ouvrière 2}{\begin{enumerate}
\item "Ceux qui souffrent ne peuvent pas se plaindre, dans cette vie-là. Seraient incompris des autres, moqués peut-être de ceux qui ne souffrent pas, considférés comme des ennuyeux par ceux qui, souffrant, ont bien assez de leur propre souffrance. Partout la même dureté que de la part des chefs, à quelques exceptions près." p138
\item "ò$\tau$o$\tau$o$\iota$ !" p159
\item "Le fait capital n'est pas la souffrance, mais l'humiliation. Là-dessus, peut-être, que Hitler base sa force." p171
\item "On a toujours besoin pou roi-même de signes extérieurs de sa propre valeur." p171
\item "Les maths supérieures ne seraient-elles pas elles aussi (cf. Chartier) un moyen de "former l'attention en tuant la réflexion" ?" p174
\item "Et ce sont les tâches misérablement payées pour lesquelles ont se fatigue le plus, parce qu'on tend de totues ses forces, jusqu'à l'extrême limite, pour ne pas couler le bon." p182
\item "Aux héros de Racine, il ne reste que le pouvoir pur, sans aucun savoir-faire." p195
\item "Aujourd'hui, serviteur absolument serviteur, sans le retournement hégélien. C'est à cause de la maîtrise des forces de la nature." p196
\item "Ne pas oublier que le sommeil est ce qu'il y a de plus nécessaire au travail." p197
\item "Tout se passe comme s'il y avait trop peu de machines, alors qu'il y en a trop." p199
\item "Je ne conçois les rapports humains que sur le plan de l'égalité ; dès lors que quelqu'un s'est mis à me traiter en inférieure, il n'y a plus à mes yeux de rapports humains possibles entre lui et moi, et je le traite à mon tour en supérieur, c'est-à-dire que je subis son pouvoir comme je subirais le froid ou la pluie." p223
\item "et j'ai acquis la conviction, fort triste pour moi, que non seulement la capacité révolutionnaire, mais plus généralement la capacité d'action de la classe ouvrière française est à peu près nulle. Je crois que les bourgeois seuls peuvent se faire illusion à ce sujet." p226
\item "Je pense que quand on a été ouvrière, il faut au moins devenir un peu paysanne, pour que l'expérience ait un sens ; il n'y a pas que les villes au monde." p255
\item "L'organisation du travail doit réaliser la combinaison de l'ordre et de la liberté." p257
\item "On chante, mais pas L'Internationale, pas la Jeune Garde ; on chante des chansons, tout simplement, et c'est très bien." p277
\item "Le pli de la passivité contracté quotidiennement pendant des années et des années ne se perd pas en quelques jours, même quelques jours si beaux." p 278
\item "L'avenir le dira ; mais cet avenir, il ne faut pas l'attendre, il faut le faire." p281
\item "Vous avez vu Les Temps Modernes, je suppose ? La machine à manger, voilà le plus beau et le plus vrai symbole de la situation des ouvriers dans l'usine." p287
\item "Voici ce que la situation présente a de plus paradoxal. Les patrons, parce qu'ils croient qu'ils n'ont plus rien à perdre, prennent le vocabulaire et l'attitude révolutionnaire. Les ouvriers, parce qu'ils croient qu'ils ont quelque chose d'assez important à perdre, prennent le vocabulaire et l'attitude conservatrice." p295
\item "Ils ont été formés à regarder le monde comme composé d'ennemis, à ne pouvoir compter sur personne" p297
\item "Le laboratoire était pour lui [Taylor] un moyen de forcer les ouvriers à donner à l'usine le maximum de leur capacité de travail." p314
\end{enumerate}
}

\Cit{La Condition Ouvrière 3}{\begin{enumerate}
\item "Rien n'est plus facile pour un industriel que d'acheter un savant, et lorsque le patron est l'Etat, rien n'est plus facile pour lui que d'imposer telle ou telle règle scientifique." p326
\item "Rien n'est pire que le mélange de la monotonie et du hasard ; ils s'aggravent l'un l'autre, du moins quand le hasard est angoissant." p334
\item "Tu étais une de ces machines de chair." p358
\item "La lutte des classes n'est pas simplement fonction des intérêts, la manière dont elle se déroule dépend en grande partie de l'état d'esprit qui règne dans tel ou tel milieu social." p363
\item "Les contremaîtres, habitués à commander brutalement, et qui avant juin n'avaient presque jamais eu besoin de persuader, se sont trouvés tout à fait désorientés ;" p364
\item "Ils doivent sentir au moins - c'est bien là un minimum - qu'ils comptent autant, en qualité d'êtres humains, que les machines et les produits usinés." p377
\item "L'imagination est toujours le tissu de la vie sociale et le moteur de l'histoire." p384
\item "L'action méthodeique, dans tous les domaines, consiste à prendre une mesure non au moment où elle doit être efficace, mais au moment où elle est possible en vue de celui où elle sera efficace. Ceux qui ne savent pas ruser ainsi avec le temps, leurs bonnes intentions sont de la nature de celles qui pavent l'enfer." p387
\item "Marx n'est pas un bon auteur pour former le jugement ; Machiavel vaut infiniment mieux." p388
\item "Imaginons à présent la semaine de trente heures établie dans toutes les usines d'automobiles du monde, ainsi qu'une cadence de travail moins rapide. Quelles catastrophes en résultera-t-il ? Pas un enfant n'aura moins de lait, pas une famille n'aura plus froid, et même, vraisemblablement, pas un patron d'usine d'automobiles n'aura une vie moins large." p393
\item "Lénine avait constitué un parti bolchevik pourréaliser la disparition progressive de l'état et régime le plus démocratique qu'on ait jamais vu sur la terre ; il avait simplement oublié de se demander si ce parti, par son fonctionnement, amènerait ce régime." p402
\item "Un moyen puissant n'est jamais puissant pour n'importe quoi, mais seulement pour réaliser ce qui résulte nécessairement de sa structure." p403
\item "Tout ce qui est beau est objet de désir,mais on ne désire pas que cela soit autre, on ne désire rien y changer, on désire cela même qui est. On regarde avec désir le ciel étoilé d'une nuit claire, et ce qu'on désire, c'est uniquement le spectacle qu'on possède." p423
\item "Le peuple a besoin de poésie comme de pain." p424
\item "Pour les travailleurs il n'y a pas d'écran. Rien ne les sépare de Dieu." p424
\item "Il serait étonnant qu'une église construite de main d'homme fût pleine de symboles et que l'univers n'en fût pas infiniment plein." p426
\item "Toutes les fois que la voix qui commande se fait entendre alors qu'un arrangement praticable pourrait y substituer le silence, c'est un mal." p433
\end{enumerate}
}

\Cit{Par-dessus bord}{
\begin{enumerate}
\item "(Vinavier) Mon objectif qui était de me faire virer a échoué." p8
\item "(Alvarez) Un bon papier c'est comme un bon service des ventes ça résiste et ça fait son travail" p20 
\item "(Dehaze) Mon père est mort la tête haute disant qu'il ne devait rien à personne" p25
\item "(Margerie) quand je suis allée sur les barricades j'ai senti que tout était pas encore foutu mais toi tu deviens tous les jours un petit peu plus manager un peu plus con" p69
\item "(Margerie) Manger travailler dormir" p69
\item "(Alex) On vous pend sec sec on vous fait avaler du sable très éclaté on reprend plus rachitique il ne reste que le squelette même plus il faut rester là-dessus presque bon vous savez" p76
\item "(Benoît) ceux d'entre vous qui n'adopteront pas la cadence eh bien ils resteront sur le quai ce n'est pas une menace c'est une constatation" p131
\item "(Battistini) et puis je m'excuse mais se regarder en train de déposer un étron au fond d'une cuvette ça n'est pas d'un intérêt transcendant" p139
\item "(Jack) Je voulais que nous en arrivions simplement à poser le postulat (Jenny) C'est ça mon intuition les enfants (Jack) Que chier est un plaisir (Jenny) Et pour beaucoup de raisons un plaisir interdit" p146
\item "(Margerie) Tu aime le caviar ? C'est tout ce qu'il y a pour dîner" p172
\item "(Jack) le consommateur est une grande bouche et un grand cul" p184
\item "(Jack) Nietzsche a dit l'homme est un animal qui fait des promesses tout se tient promettre c'est être constipé avec le passé" p184
\item "(Peyre) Ces dizaines de millions de mètres carrés de papier qui passent entre les mains des consommateurs tout cet espace inutile (Benoît) Oui on pourrait (Battistini) Y imprimer de petits textes instructifs (Peyre) Une encyclopédie sans début ni fin (Battistini) Infinie formez-vous (Peyre) Informez-vous au hasard de vos selles" p201-202
\item "(Bachevsky) Quand dois-je partir ? (Grangier) Nous sommes vendredi (Benoît) Ce soir madame Bachevsky" p208
\item "(Young) Vous entrez dans une grande famille" p250
\end{enumerate}
}

\Cit{Géorgiques}{
\begin{enumerate}
\item "Et toi enfin, qui dois un jour prendre place dans les conseils des dieux à un titre qu'on ignore, veux-tu, César, visiter les villes et prendre soin des terres et voir le vaste univers t'accueillir comme l'auteur des moissons et le maître des saisons, en te ceignant les tempes du myrte maternel ?" p39
\item "Et cependant, en dépit de tout ce mal que les hommes et les boeufs se sont donné pour retourner la terre, ils ont encore à craindre l'oie vorace, les grues du Strymon, l'endive aux fibres amères et les méfaits de l'ombre." p45
\item "son but était, en exerçant le besoin, de créer peu à peu les différents arts, de faire chercher dans les sillons l'herbe du blé et jaillire du sein du caillou le feu qu'il recèle." p46 
\item "Je puis te rappeler une foule de préceptes des anciens, si tu n'y répugnes pas et ne dédaignes pas de connaître de menus détails. p49
\item "Beaucoup de travaux nous sont rendus plus faciles par la fraîcheur de la nuit ou par la rosée dont, au lever du soleil, l'Aurore humecte les terres." p55
\item "Mets-toi nu pour labourer, mets-toi nu pour semer : l'hiver, le cultivateur se repose." p56
\item "Avant tout, honore les dieux, et offre à la grande Cérès un sacrifice annuel en accomplissant les rites sur de gras herbages, quand le déclin de l'extrême hiver fait déjà place au printemps serein." p58
\item " Si tu observes le soleil dévorant et les phases successives de la lune, jamais le temps du lendemain ne te trompera, ni jamais tu ne te laisseras prendre aux pièges d'une nuit sereine." p63
\item "Salut, grande mère de récoltes, terre de Saturne, grande mère de héros ! C'est pour toit que j'entreprends de célébrer l'art antique qui a fait ta gloire, osant rouvrir les fontaines sacrées, et que je chante le poème d'Ascra par les villes romaines." p84
\item "Non, ce ne fut pas d'autres jours - je le croirais volontiers - qui éclairèrent le monde naissant à son origine première, ni une autre continuité de température : c'était le printemps, le printemps qui régnait sur l'immense univers" p93
\item "Mais fortuné aussi celui qui connaît les dieux champêtres, et Pan, et le vieux Silvain, et les Nymphes soeurs !" p102
\item "C'est moi qui le premier, si ma vie est assez longue, ferait descendre les Muses du sommet Aonien pour les conduire avec moi dans ma patrie ;" p110
\item "Les plus beaux jours de l'âge des malheureux mortels sont les premiers à fuir : à leur place viennent les maladies et la triste vieillesse, puis les souffrances, et l'inclémence de la dure mort nous prend." p115
\item "L'ouragan qui déchaîne l'orage s'abat moins fréquemment sur la mer que les épidémies sur les bêtes, et les maladies n'attaquent pas quelques individus isolés, mais enlèvent tout à coup des parcs d'été tout entiers, l'espoir du troupeau et le troupeau en même temps, et toute la race depuis son origine." p137
\item "Mince est le sujet, mais non mince la gloire, si des divinités jalouses laissent le poète chanter et si Apollon exauce ses voeux." p145
\item "Quand tu auras fait quitter le champ de bataille aux deux chefs, livre à la mort celui qui t'a paru le plus faible, afin qu'il ne soit pas un fardeau inutile : laisse le meilleur régner seul dans sa cour." p150
\item "de même, s'il est permis de comparer les petites choses aux grandes, les abeilles de Cécrops sont tourmentées d'un désir inné d'amasser, chacune dans son emploi." p155
\item "Tant que ce roi est sauf, elles n'ont toutes qu'une seule âme ; perdu, elles rompent le pact, pillent les magasins de miel, brisent les claies des rayons." p157
\end{enumerate}
}



\end{document}