\documentclass[a4paper,12pt]{book}
\usepackage{ae}
\usepackage{aeguill}
\usepackage{amsthm}
\usepackage[utf8]{inputenc}
\usepackage[T1]{fontenc}
\usepackage[french]{babel}
\usepackage[utf8]{inputenc}
\usepackage{graphicx}
\usepackage{hyperref}
\usepackage{xcolor}
\usepackage[left=1cm, right= 1cm, top=2cm, bottom = 2cm]{geometry}
\usepackage{array,multirow}
\usepackage{amsmath,amsthm,amssymb}
\usepackage{amsfonts}
\usepackage{stmaryrd}
\usepackage{tcolorbox}
\usepackage{lmodern}



\newcommand{\Def}[2]{\begin{tcolorbox}[sharp corners, colback=white,colframe=red!90!black!75, title=Définition : #1]#2\end{tcolorbox}}
\newcommand{\Thr}[2]{\begin{tcolorbox}[sharp corners, colback=white,colframe=red!90!black!75, title=Théorème : #1]#2\end{tcolorbox}}
\newcommand{\Prop}[2]{\begin{tcolorbox}[sharp corners, colback=white,colframe=red!90!black!75, title=Proposition : #1]#2\end{tcolorbox}}
\newcommand{\Pre}[1]{\begin{tcolorbox}[sharp corners, colback=white,colframe=green!60!green!30!black!75, title=Preuve]#1\end{tcolorbox}}

\newcommand{\Meth}[2]{\begin{tcolorbox}[colback=white,colframe=green!60!green!30!black!75, title=Méthode :  #1]#2\end{tcolorbox}}
\newtheorem{Exe}{Exemple}[section]
\newtheorem{Exes}{Exemples}[section]
\newtheorem{Rem}{Remarque}[section]
\newtheorem{Rems}{Remarques}[section]

\def\R{\mathbb{R}}
\def\D{\mathbb{D}}
\def\C{\mathbb{C}}
\def\Q{\mathbb{Q}}
\def\N{\mathbb{N}}
\def\Z{\mathbb{Z}}
\def\K{\mathbb{K}}

\renewcommand{\thechapter}{\Roman{chapter}}
\renewcommand{\thesection}{\Roman{section}}
\renewcommand{\thesubsection}{\Roman{section}.\arabic{subsection}}
\renewcommand{\thesubsubsection}{\Roman{section}.\arabic{subsection}.\Alph{subsubsection}}

\title{Cours de français}
\author{Batraciémy}

\begin{document}
guy.barthelemy1\texttt{@}free.fr
\tableofcontents

\chapter{Méthodes}
\par Les estrades ont disparu d'abord dans un mouvement qui n'aurait aucun sens, pour mettre les élèves et les professeurs sur un pied d'égalité, et aussi pour un besoin sécuritaire. Mais c'est con parcer que ça veut dire qu'on ne voit pas d'étudiants.
\par Le cours sera constitué d'abord de plusieurs résumés sur des textes qui permettront de baliser le programme, puis quelques cours sur Weil, puis quelques cours sur Virgile, puis quelques cours sur Vinavier.
\par Les épreuves associent le plus souvent un résumé et une dissertation en 4H. D'autres concours (EDEC : école de commerce) imposent par exemple une synthèse avec plus de documents. Après les résultats d'admissibilité, il faudra aller voir quelles épreuves on passera à l'orale, qui vont d'un entretien à un test de culture générale.
\section{Résumé}
Qu'est-ce qu'on attend d'un résumé ? Il faut que le vocabulaire du texte soit compris (oraliser le texte dans sa tête rend les choses plus simples, histoire d'avoir une meilleure interprétation du texte), ainsi que sa syntaxe. On demande ensuite à réduire le texte tout en en gardant la substance, capacité très utile dans la vie (fable du ministre de l'économie et de coco, chef de cabinet à apprendre, visiblement). Il faut faire appel à des synonymes, sauf pour certains mots techniques (comme génétique). Tout ce qui peut être remplacé doit l'être, les tournures de phrases ne doivent pas être copiées. Dans le texte, il faut repérer le circuit et la logique des arguments. Il faut trouver puis restituer tous les arguments du texte, de manière plus rapide. C'est là qu'on doit mettre en oeuvre sa capacité logique et son vocabulaire.
\par Il ne faut pas faire trop d'alinéas : un paragraphe, c'est une \underline{unité de sens} logique, et simplement juxtaposer des unités n'est pas un propos organisé. Dans un résumé, on se doit d'organiser et de développer les idées présentes. Un résumé, ce n'est pas une juxtaposition de dix morceaux d'une ligne et demie. On transforme le texte, et on ne reste pas sur sa structure syntaxique, mais sur sa structure logique. Sur les paragraphes, il ne faut pas faire des paragraphes de 25 lignes, c'est bien trop peu lisible. Au lieu de faire de longs paragraphes, on les subdivise. La lecture d'un paragraphe dépend du format.  
\par Sur la synonymie, il ne faut pas faire apparaître un langage trop étroit, et surtout ne pas mettre des synonymes qui changent le sens du texte. C'est la faute de notre entre-nous sociale brouillasseux et laxiste. Beaucoup d'informations se perdent comparé à la norme d'orthodoxie et de clarté intellectuelle. Nous autres serions habitués à nous satisfaire d'une imprécision linguistique (utiliser le projet Voltaire, ou bien d'utiliser "Améliorez votre style" [éditions Hatier], "S'exprimer avec logique" [éditions Hatier] et Savoir Rédiger [Larousse]). Le mieux est de prendre quelques demi-heures dans la semaine pour les travailler, et lire des articles de "philosophie magazine", les "pages débat"/décryptage du Monde et de Courrier International, le tout en faisant une lecture pas à pas, en analysant ce qu'on écrit. Le résumé est un exercice difficile, on le sait parce qu'il a été supprimé du bac et que le bac est donné à tout le monde.
\section{Dissertation}
Sur la dissertation : on attend un discours argumentatif clair sur les oeuvres au programme, le sujet étant un prétexte pour faire réfléchir, la réponse au sujet n'étant pas l'intérêt de la dissertation. Il faut fournir un discours interprétatif (entre 2 et 5 pourcents de l'écrit peuvent mentionner d'autres oeuvres, notamment l'introduction). Il faut connaître les trois oeuvres et pouvoir les invoquer ensemble facilement. "Nous nous demandions si [sujet] et nous avons vu que [première partie], que [deuxième partie] et que [troisième partie]" est la phrase par excellence pour faire la synthèse des parties juste avant la conclusion et la réponse au sujet "de tout cela il ressort que", juste avant l'ouverture, qu'on peut oublier si l'on n'a rien à dire.
\par L'histoire de la pyramide : une disserte est une pyramide, son sommet est à atteindre par des briques différentes, le tout sur le territoire que vous laisse le sujet, chaque brique étant une analyse du cours. C'est une épreuve de vitesse, qui demande de rapidement pouvoir plonger dans le cours et dans les textes. L'impression que les correcteurs cherchent, c'est qu'il y a une logique suivie, tirée en avant comme par un ressort par la copie. L'inverse de ça est la copie Marivaux, qui ajoute de tout et de rien dans toutes les directions, qui ne s'arrête jamais alors qu'elle le devrait.
\section{Sur le cours}
Le cours est un ensemble d'exposés en trois sous-parties, pour se rapprocher de la dissertation. 
\par Blog du professeur : guy barthélémy littérature, hautetfort, page de méthodologie
\section{Sur les khôlles}
Il y aura deux types de khôlles : le premier est une lecture de texte, préparée en une demi-heure, où on analyse un texte avant de réfléchir à la petite question posée dans le prolongement du texte. L'important est de montrer \textbf{linéairement} la logique du texte d'abord, puis ensuite de faire un exposé avec un début, un milieu et une fin, toujours logique. La qualité de l'expression est extrêmement importante dans cet exposé. Les khôlles de ce genre seront faites par madame Lacroix, où un groupe de deux passera sur le même texte en écoutant l'autre, avant de faire une reprise de l'exposé à noter pour les deux.
\par La deuxième série de khôlle c'est l'entretien de personnalité, dont s'occupe le professeur. On vient à quatre en même temps, et on passe autant en tant que bourreau que victime. Il faut, pour préparer l'entretien, d'une part avoir un état d'esprit citoyen (suivre l'actualité et avoir des opinions) et montrer sa curiosité scientifique (savoir ce qu'est un soleil artificiel [nom étrange qu'on donne à des appareils qui savent faire de la fusion], etc.).
\par Salles de khôlle : BC 02 005 pour Lacroix et B203 pour Barthélémy
\section{La prise de note}
Qu'est-ce que la prise de note ? L'image des briques aurait un rapport avec ça. Il faut être autonome dans sa gestion de l'espace de la page. Divers ratages s'expliqueraient par une déficience en termes de prise de note des candidats, celleux-ci confondant l'aide-mémoire et la prise de notes. Le premier est construit en lisant plusieurs fois un corpus, avant de noter les énoncés essentiels et par association générer un propos. Cependant, l'aide-mémoire n'est exploitable que pour un seul individu, qui finit par cesser de l'être quand la mémoire disparaît. La prise de note quant à elle, est incomprise aussi. Ce qu'on pense souvent sur la prise de note, c'est qu'on devrait capter les "idées essentielles", chose absurde. La prise de note sert à consigner la cohérence et l'exhaustivité du propos. La cohérence, c'est faire tenir ensemble quelque chose de manière productive. Une prise de note n'est pas constituée de petits paragraphes juxtaposés, il faut une analyse développée. Tout ce qui est dit par le professeur a de la valeur en soit.

\chapter{Florilège}
Dans le film de Truffaut (qui passe au Méliès dans la semaine), ce propos sur le mode humoristique reprend une atninomie banale : celle entre l'hédonisme (conception de l'existence orientée vers le plaisir délicat) et le travail (forme de contrainte et de souffrance, voir l'étymologie de travail, souvent associée au tripalium, instrument de supplice ; le travail aurait le danger de devenir un instrument de souffrance).
\section{Chapitre 3 de la Génèse}
Adam et Eve sont bannis du jardin d'Eden pour avoir mangé le fruit de la connaissance, c'est une des parties les plus connues de la Bible. (La traduction n'est pas forcément assez solennelle d'après notre professeur.) Le serpent a convaincu Eve de manger le fruit de l'arbre de la connaissance, le serpent argumentant que Dieu craignait l'homme capable de la connaissance. Ayant découvert la nudité, Eve et Adam (qu'elle a convaincu) se cachent, ce qui permet à Dieu de comprendre que les deux ont désobéi à son ordre.
\par C'est un mythe étiologique. "déréliction" : situation où on est privé de quelque chose de fondamental, dans la religion à l'origine le fait d'être privé de l'accès au jardin d'Eden et de sa paix. Ce texte inaugure une grande ambiguïté reliée au concept de travail : les hommes d'église vont souvent mettre l'accent dans leurs prêches soit sur le fait que le travail vient d'une malédiction, associée à d'autres (comme la mort et la douleur de la grossesse), soit sur le fait que le travail soit associé à la transformation (si l'homme bloqué dans la nudité était resté enfermé dans le jardin d'Eden, dans un lieu où la nourriture était à portée de main, il n'y aurait pas eu d'histoire de l'humanité). Dans le monde sans malédiction, il n'y avait qu'une éternité de répétition. L'éviction des hommes est une dramatisation de la condition humaine, qui mène immédiatement à la violence fratricide, de Cain à Abel, au moment où les hommes inventent les choses essentielles comme l'agriculture et l'élevage.
\par Dans des théologies optimistes comme celle des jésuites (XVIIe siècle, opposée aux jansénistes qui mettent l'accent sur la déchéance de l'humanité), dans des sociétés où la déréliction est vue comme une partie de l'histoire, juste après la recherche du paradis terrestre dans l'âge d'or de l'exploration du monde par l'occident. Cette théologie optimiste, donc, dit que par sa capacité à agir au sens le plus large, l'homme rend hommage à Dieu, et qu'il doit s'efforcer de faire de grandes choses au nom de Dieu qui lui a donné toutes ces capacités. L'ambiguïté du travail est dans la double-interprétation de ce mythe, où le travail est obligatoire.

\section{Extrait de Madame Bovary, Flaubert}
Les comices agricoles sont des fêtes organisées pour célébrer la prospérité agricole, où plusieurs personnes sont récompensées pour leurs mérites. On donne une médaille à une vieille servante de ferme. Un béguin est une sorte de coiffe. "Monacal" : adjectif associé au substantif moine. "Atonie" : inertie mentale, absence de vivacité. 
\par Le dernier mot du texte est "servitude", mot étymologiquement proche d'esclavage. Ce texte nous parle d'un individu qui est une illustration vivante de l'exploitation et de l'aliénation menant à la déshumanisation. Cette femme a usé son corps dans l'accomplissement de tâches pour d'autres, des dominants sociaux, les "bourgeois épanouis". Elle était rémunérée de manière à travailler énormément, tout en faisant des tâches qui détruisaient son corps, ses mains en témoignant, à jamais abîmées par les travaux. L'aliénation, c'est le fait de devenir étranger à soi-même du fait de l'exploitation. Cette aliénation est traduite par le fait qu'elle ne soit qu'un instrument, ce qui se manifeste dans ses mains ouvertes comme des pinces au service de ses dominants, et se voit aussi par l'évidement intellectuel et affectif de sa conscience. C'est le portrait d'une femme déshumanisée par son travail et par la relation sociale d'exploitation sous-jacente. Flaubert la décrit comme un animal, mais la décrit avec de la compassion, pas du mépris.
\par Au même moment, son ironie met en cause ces "bourgeois épanouis", qui ont forcé la déshumanisation de cette femme, qui vont cyniquement récompenser l'acceptation de la servitude. Ce comportement de bon esclave est récompensé, et c'est là l'ironie de Flaubert.

\section{Extrait de La Mère aux monstres, Maupassant}
On se fait souvent l'image d'un Maupassant réaliste sans grande envergure, avec un pessimisme très radical, dû à Schopenhauer. Il y a cependant derrière ses histoires un réalisme très identifié. Il évoque une fille de femre, quelqu'un tout au bas de l'échelle, qui se voit affligée de l'indignité sociale. On voit la pression de la morale bourgeoise et de sa puissance normative. Il lui est arrivé quelque chose de très banal : elle a eu accès à la sexualité en-dehors du mariage. C'était surtout quelque chose qui arrivait aux domestiques de la part de leurs maîtres ; dès qu'elles finissaient dans un état "intéressant", on les virait. Elle a alors deux choix : dissimuler sa grossesse ou l'admettre et être mise à la porte.
\par Pour cacher sa grossesse, elle se fabrique un "corset" : usage dominant dans le monde bourgeois (marché matrimonial : capital érotique des femmes lié à l'étroitesse de la taille ; croyance médicinale : corps des femmes trop faibles, nécessitant un corset, mais c'est une prophétie auto-réalisatrice, où les femmes à la taille écrasée sont incapable de vivre sans corset) mais ici, c'est quelqu'un de très loin de ces échelles sociales qui en porte un. Elle a intériorisé sa situation (dominée/pécheresse) de telle manière que son corps en est affectée. Elle ne donne ainsi pas naissance à un enfant normal mais à un monstre. Elle est donc surnommée "la Diable", séparée de l'humanité. Elle va intérioriser un nouvel aspect de la bourgeoisie : il faut gagner de l'argent par tous les moyens, se trouver une niche. Elle va donc devenir une sorte d'entrepreneur en monstruosité, et fabriquer des monstres différents les uns des autres avec son système de corset. Elle va bien réussir ; si ces sommes ne nous disent rien, l'auteur décrit sa situation finale comme importante. Sa seule possibilité une fois le premier monstre né, c'est de devenir riche grâce à ça. C'est une allégorie d'une société dominée par la fascination pour l'argent.
\par "La bourgeoisie a tout noyé dans les eaux glacées de l'intérêt" disaient Marx et Engels. Elle n'a plus rien d'humain, pas d'amour maternel. La seule raison pour laquelle elle n'a pas tué son monstre, c'est qu'elle craignait la punition de la justice. Il y a un jeu de mot entre le travail d'un accouchement (moment où la femme commence à accoucher) et le travail qu'elle trouve à mettre au monde des monstres. C'est une anti-mère, qui l'est devenue dans une situation particulière. Elle paie de sa personne, et ça lui rapport au prix de son humanité. Il y a un parallèle fait dans la nouvelle avec une autre femme riche qui accouche de monstres à cause de la pression sociale des corsets dans la bourgeoisie.

\section{Extrait de Germinal, Zola}
C'est un des premiers extraits de Germinal, qui est là pour donner le ton. Dans la conception qu'on a de lui, Zola est un auteur qui décrit les conditions de vie des mineurs avant tout. On trouve cependant un souffle bien peu réaliste, bien lisible dans l'extrait, avec les deux descriptions du Voreux, un trou de charbon qui sortait du parler des mineurs, pas du français académique. Voreux est un mot qui évoque l'idée de dévorer, une entité vorace. Dans la description, on le décrit comme une "bête méchante", "gêné par sa digestion pénible de chair humaine". C'est une manière très forte et marquante de dire que le capitalisme exploite des gens en les dévorant par les mines. Zola nous décrit un monde mort, proche d'un âge de fer d'Hésiode (marqué par la dévastation et la guerre), qui n'est plus éclairé par la clarté féconde du soleil.
\par On rencontre un personnage emblématique de ce monde : Bonnemort. Il a failli mourir par trois fois dans les puits de mine, et donc par ironie (l'humour est une manière de se défendre pour les dominés), on l'a surnommé ainsi. Il est décrit comme dégénéré, une grande crainte du XIXe. Zola est exploité par l'hérédité, par la pureté de la "race", et la dégénérescence qui s'empare d'elle. On peut repenser à l'enquête d'Engels à Londres, qui découvre des gens vivant dans des caves, épuisés. À l'époque on pense que ces caractéristiques altérées par le travail passeront leurs traits à leurs enfants. Le corps est endommagé, et le charbon transforme tout. Bonnemort crache du charbon, à foirce de l'avoir absorbé toute sa vie. C'est une autre manière de dire que les personnes sont dévorées par la mine. On n'est pas dans une société humaniste, au contraire on le déshumanise, le rapprochant de son cheval. Il a été conduit une assiduité qui lui a permis de survivre. Il est très fier de sa généalogie, signe de distinction de la noblesse par excellence, source d'ironie car généalogie de la faim et de la pauvreté, ainsi que de la spoliation (le grand-père de Bonnemort a trouvé la première veine de charbon et a vécu dans la pauvreté) et de la mort, le "sang bu par un rocher".
\par La mine qui boit le sang est une allégorie des gens qui dirent du profit de la mine. Hennebeau, directeur, ne reste qu'un salarié, d'après Bonnemort. Bonnemort dit que la mine appartient à "des gens", rien de plus. Dans un roman nommé \textit{L'Argent}, Zola décrit le capitalisme de manière très technique. Il sait très bien qu'on a besoin de créer une société par actions pour commencer une mine, ce qui demande des actionnaires qui espèrent des dividendes. Ce sont eux, les exploitants invisibles décrits par Zola. Si un salaire permet à peine de vivre, on ne peut investir. Il faut une marge. Pour que les profits soient faits, il faut exploiter les ouvriers.
\par Le roman parle des accidents (gaz, coups de grisou, etc.) avec beaucoup d'insistance sur le boisement des galeries qui empêche l'effondrement. Les mineurs qui les creusent étant payés à la quantité de charbon ramené, vont plus vite et n'étaient pas très bien leurs tunnels. Ils sont ensuite punis si quelque chose se passe mal, malgré les conditions atroces qu'ils doivent subir. C'est une exploitation démente, un mécanisme forcené dans lequel le travail va au mieux déshumaniser les individus (\textit{Portrait de l'homme mort}) et au pire les tuer.

\section{Extraits sur les camps de la mort}
Une rapide définition de l'esclavage : l'esclave n'est pas libre, dans le sens où il appartient à une autre personne. Dans notre imaginaire, les esclaves sont soit des prisonniers de guerre, soit des débiteurs qui n'ont pas pu rembourser, soit des condamnés, soit des individus vus comme d'une nature inférieure. Dans cet imaginaire, on accorde souvent un sens de tavail gratuit, bien que ça ne soit pas le cas partout (voir l'empire ottoman, où l'esclave a le droit d'accumuler un peu de pécule pour s'auto-racheter). Notre imaginaire vient de l'esclavage dans les champs de coton et de sucre en Amérique, avec l'idée qu'ils n'aient pas de droits. Notre imaginaire de l'esclavage et notre usage de ce mot sont déterminés par cette idée. Aujourd'hui, les immigrés qui se font prendre leur passeport vivent dans des situations similaires, renforçant cette image de l'esclavage. L'esclavage est le contraire du travail, l'esclave c'est l'anti-travailleur.
\par "Le travail rend libre" est une évidence pour un démocrate, mais était antithétique avec l'esclavage et l'extermination qui venait dans les camps. Pour les textes de Lévi, il faut rappeler que c'est un intellectuel partisan qui se retrouve arrêté et envoyé dans un camp. Il redécouvre ce qu'est un travail normal après un temps dans les camps, et est profondément surpris par le fait qu'on lui demande si peu. Une autre dimension du "travail rend libre" est celle du camp de rééducation.
\par Le texte d'Himmler est pour nous stupéfiant. Mais il faut se souvenir que des milliers de personnes ont adhéré à ces idées philosophiquement et logiquement racistes. Cette vue, qui considère que certaines personnes ne sont pas humaines et doivent être éliminées. On voit le sadisme délibéré du nazisme, qui consiste soit à faire accomplir des tâches profondément inutiles et épuisantes. Avant d'être mis à mort, il faut être humilié, en imposant aux gens une vision d'eux mêmes de gens inutiles.
\par Le troisième texte dit que du point de vue des nazis, les tortures des humains dans les camps sont justifiées par le besoin des soldats allemands de ne pas avoir mal aux pieds. On voit bien le délire nazi, débauche d'énergie inutile (mettre à mort serait plus simple, mais on leur impose des tortures quand même).

\section{Extraits de La Société contre l'Etat, Pierre Clastres}
À partir des grandes découvertes du XVIIIe siècle, la rencontre d'autres sociétés va mener à la pensée des sociétées organisées en deux pôles, d'un côté des sociétés très organisées, où la division du travail est poussée très loin (voir l'Empire de Chine), et de l'autre côté des groupes de petite taille, pas organisés autour d'un Etat, ou la culture matérielle est plus réduite. La vision que reprend Clastres dans le contexte historique des années 70, c'est celle de voir dans ce second pôle une sorte de sagesse instinctive qui mène les gens à refuser le processus d'accumulation. Ce refus est pour lui garantie d'une sorte de bonheur caractérisé en premier lieu par le très petit temps de travail.
\par Le peu de temps dédié au travail permet alors de dédier plus de temps au plaisir. On retrouve ici le paradigme qui associe et oppose le temps de travail et le temps de liberté. Dans notre situation, face au péril écologique, certaines personnes parlent de décroissance, disant qu'il faudrait limiter la culture matérielle, où nous ne serions pas inciter à renouveler nos objets. Lévi-Strauss a aussi travaillé sur les mêmes groupes, et dit qu'une famille tentait de tout garder dans un même sac, tandis que la société actuelle a besoin de camions pour un déménagement. Clastres raisonne selon la conviction anarchiste selon laquelle l'Etat signifie le malheur, et fait l'éloge des personnes qui ont compris que le travail est une malédiction.
\par Ce que ce texte ne dit pas, c'est que la notion d'économie de subsistance fait débat ; elle est parfois décrite comme une économie toujours précaire où on cherche à tout prix à survivre en produisant le plus possible, toujours comparée à une société d'abondance (où on recherche des choses futiles). C'est très relatif, un ouvrier dans une société d'abondance comme la nôtre dans les années 30, pouvait ne pas avoir assez pour manger le soir. Clastres ne montre cependant pas une humanité qui fait toujours face à la pénurie, mais une humanité heureuse, organisée autour du loisir et pas du travail, l'ayant exorcisé.

\section{Extraits de Troubles dans le Travail, Dujarier}
Le premier texte parle de fait social total, théorisé par Mauss et Durkheim, fondateurs de la sociologie française. C'est un sujet qui relie tous les aspects de la société, comme l'alimentation, reliée à la médecine, au statut social et à l'économie internationale par exemple. Le travail dans une société comme la nôtre joue un rôle de fait social total, et ce que ce texte met en avant, c'est que c'est à partir du travail (et parfois de son absence) qu'on définit la position sociale des individus. Celui qui ne travaille pas est défini négativement quant au concept de travail. C'est la première définition du statut de l'individu, que ça soit sous une forme de dignité pour un retraité ou sous une forme de scandale pour celui qui n'a pas d'emploi. Enfin, la sociologue se place sur le point subjectif et psychologique en parlant de ce que ressentent les personnes à qui on demande leur travail et qui doivent répondre qu'elles ne travaillent pas.
\par Le deuxième texte ramène à des débats contemporains. Certaines activités coûteuses en temps et très difficiles psychiquement, comme les soins palliatifs, ne font l'effet d'une rémunération. Le texte prend le cas limite de la grossesse. En l'état actuel, elle n'est pas rémunérée, mais le congé maternité signifie quand même que la collectivité nationale reconnaît son devoir de financer la période d'inactivité avant et après l'accouchement. Et dans un monde moderne, les deux parents préparant un futur membre de la collectivité, le père a aussi le droit d'un congé paternnité. Le domaine du "care" (soins apporté par les enfants à leurs parents vieillissant notamment) n'est pas rémunéré malgré la quantité de travail demandée. La justification pour la non-rémunération est de ne pas rémunérer quelque chose de naturel. On peut rapprocher ça des intégrismes qui infériorisent la femme en lui disant de se concentrer sur la maternité tandis que l'homme a le droit d'entrer dans la sphère sociale. C'est l'argumentaire des intégrismes mais pas que.
\par On invoque la naturalité et l'évidence des relations parents-enfants : les fils mais surtout les filles doivent s'occuper de leurs vieux parents incapables. Et si on se met à rémunérer la piété filiale, ça serait insérer l'argent dans des relations d'amour. Et ça impliquerait beaucoup de changements économiques. Dans les pays comme l'Allemagne ou le Portugal qui considèrent que les femmes doivent se consacrer à leurs enfants et ne développent pas de crèches ni de systèmes pour permettre aux femmes de travailler, faisant que les femmes décident de faire moins d'enfants. En France d'un autre côté, on a une politique plus nataliste, qui donne plus de possibilités pour concilier une vie professionnelle avec un travail aux femmes. Souvent, il manque en plus quelques trimestres à beaucoup de femmes pour s'être consacrées à leurs enfants.
\par La catégorie du travail n'est donc pas aussi simple qu'elle en a l'air. 

\section{Extraits de L'hyperaction dans l'emploi comme esquive du travail psychique issu de Troubles dans le Travail, Dujarier}
Dans le début du texte, la sociologue explore deux versants de la souffrance au travail, d'abord celle de l'ouvrier taylorisé, montré par Chaplin dans les temps modernes, quand il devient fou. Ses stratégies de défense consistent alors à endormir la conscience, d'abord parce que c'est difficilement supportable, et surtout parce que la conscience ralentit la cadence. Plus la cadence est forte, moins il pense. Il souffre moins et travaille plus.
\par Le deuxième cas traité est proche de la vie professionnelle qui nous attend. On leur demande de manifester des compétences dans des causes qui manquent de sens. Ils se retrouvent dans un cas de mauvaise foi Sartrienne, où on travaille énormément et tente de se convaincre qu'on a raison de faire quelque chose de profondément inutile. Le sens du travail a des aspects paradoxaux. Par exemple dans le monde du professorat, le métier perd du sens car de plus en plus technocratisé, de plus en plus soumis aux parents. Mais pourtant c'est un métier porteur de sens par excellence, visant à créer des individus au lieu de simples employés. Mais les problèmes avec le ministère, avec l'administration en général et le salaire misérable font que beaucoup ne se tournent pas vers le corps professoral. Un moyen d'exciser la souffrance du manque de sens est de se réfugier dans le travail ("si tu as des ennuis, mets des chaussures trop petites").
\par Sur la souffrance au travail : dans une société qui ne se contente plus de prendre en compte les pathologies physiques, on prend de plus en plus en compte la souffrance. Si on est cadre, après un conflit avec un supérieur, on sera placardisé, notre valeur devenant de plus en plus faible et sans aucun travail demandé, ce qui cause des problèmes d'estime de soi. Bien sûr le harcèlement sexuel et autres problèmes sont aussi pris au sérieux, puisqu'on connaît des cas où les politiques managériales ont mené à des suicides, à EDF par exemple.

\section{La photo de Gandhi et du rouet}
Sur le symbole du rouet et l'auto-production, il faut comprendre ça comme un conflit asymétrique. Obtenir l'indépendance par la lutte armée est difficile, mais il propose une autre voie : si les indiens n'achètent plus ce que les anglais leur fournissent, l'intérêt économique disparaît, et donc il y a une raison de leur accorder l'indépendance. Ghandi promeut alors l'auto-suffisance. L'asymétrie est symbolisée par un rouet, technologiquement peu compliqué, rien à voir avec une filature industrielle. C'est une manière de promouvoir l'autonomie, et l'artisanat, opposé à la filature qu'on n'apprend pas à maîtriser. Gandhi, en tenue traditionnelle et s'occupant de son rouet sans vraiment y prêter attention, présente un opposé absolu de l'aliénation de la filature.
\par Il y a une forte dimension économique et symbolique dans ce boycott, qui peut être comparée avec le boycott d'une marque d'oranges d'Afrique du Sud, suite à une campagne avec des affiches montrant une main blanche écrasant un enfant noir dans un pressoir à oranges. 


\section{Sujet d'entraînement}
Penser le travail, est-ce penser l'homme ?
\par Travail : rémunéré ? remplissant la journée ?
\par Penser le travail = réflection sur le travail historique, sociologique
\par Penser l'homme : conceptualiser la société, analyser la vie humaine, rendre compte de la place de l'homme dans l'univers
\par Partie 1 : Le travail est un fait social total\begin{itemize}
    \item On accorde une portée morale au travail, de la religion au 
    \item L'être humain actuel est défini devant ses pairs par son travail
    \item On voit le travail comme une force aliénante, qui peut mener à une dégénérescence de l'humanité ou qui peut punir une dégénérescence de l'humanité (Germinal)
\end{itemize}
\par Partie 2 : Le travail en tant que force aliénante \begin{itemize}
    \item Le travail en tant que destruction et affaiblissement des liens familiaux (Mère aux monstres)
    \item Le travail en tant que chose à devoir gérer en même temps que d'autres obligations, qui demandent un travail dans d'autres domaines
\end{itemize}
\par Partie 3 : La détente en tant que force dans la construction sociale\begin{itemize}
    \item Le travail comme opposé à la détente
\end{itemize}


Face à un sujet, une erreur essentielle à ne pas se commettre est de se demander si on est d'accord, puisque ça instaure un blocage. Ensuite, il faut écrire pour bien se concentrer, maintenir l'intensité de l'activité intellectuelle et la canaliser dans un but précis : le sujet. L'épreuve standard dure quatre heures, elle va très vite, demandant une extrême concentration.
\par Il faut essayer d'analyser le sujet : prendre le terme le plus important (ici, l'homme) et lui associer et y distinguer toutes les dimensions qui y coexistent. On en tire trois parties rudimentaires : le travail permet de penser la dimension individuelle et métaphysique de l'humain (Genèse : homme victime malédiction dont le travail est un symbole mais permet gagner dignité humaine ; le travail comme souffrance et aliénation dans un monde injuste, le travail peut être destructeur ; le travail comme idéal humaniste et instrument par lequel l'individu trouve sa place et en tire sa liberté), le travail permet de penser la société humaine (le projet d'une société du travail et des exigences sociales où chacun contribue en travaillant : Macron ; les sociétés du chômage comme en crise, avec l'alourdissement des impôts et des tensions sociales ; ) et est-ce que le travail est suffisant ? (paradoxe que rien n'échappe à la perspective du travail, car le temps libre est défini par l'absence de travail, les âges de la vie se définissent par rapport au travail)


\section{Second sujet d'entraînement : résumé}
Il faut d'abord présenter le texte en une ou deux phrases, en faire un sommaire : Dans ce texte, Louis Blanc affirme que l'homme est dôté d'aptitudes naturelles qui le prédisposent à une fonction et qu'une société dans laquelle le travail serait aligné sur ces aptitudes serait une société où les individus travailleraient avec plaisir, donc rendraient à la société ce qu'elle leur donne et ça serait aussi un monde sans paresseux.
\par Ensuite, il faut parcourir le texte. Dans le premier paragraphe, on énonce le double-axiome : avec les facultés, chacun reçoit des besoins ; puis la deuxième partie : à chacun selon ses besoins. On a un "or" qui permet à Blanc de mentionner un corollaire à son premier énoncé, qui est le deuxième axiome (à chacun selon ses besoins), qui est que la société ne devrait pas être un obstacle à ces besoins. La société est un ensemble de principes politiques. On a ensuite une question rhétorique, qui permet d'introduire le deuxième énoncé. C'est vu comme question de justice. "Là est le droit" sert à accentuer la différence entre le droit de fait et le véritable droit, qui n'est pas réel mais qui est le seul juste.
\par Blanc décrédibilise ensuite ses adversaires, en les affirmant superficiels, qui l'accusent d'"Utopie". Le sens actuel d'utopie est de qualifier un projet qui ne repecte pas les contraintes réelles. Mais c'est une définition faible du mot, là où le mot originel signifiait projet d'une société idéale. Souvent, les utopies impliquent la fin de la liberté individuelle (cf Huxley, Brave New World et 1984). Mais dans le contexte de Blanc, les deux sens se mêlent. À son époque, il y a beaucoup de projets utopistes, comme dans les usines Godin, projets où les ouvriers étaient beaucoup mieux traités. Les deux sens sont présents, et sont faits par des hommes (et pas des femmes) superficiels.
\par Sur la nature du besoin, son argumentaire est assez simpliste (on ne s'arrête pas de faire quelque chose juste parce que notre besoin est rassasié), et il mentionne des besoins morbides sans les développer. Blanc croit en une nature en toutes choses mesurées. Pour lui, conventionnel signifie social ; les obstacles conventionnels sont les règles sociales de la société. Il donne ensuite un exemple de personnes qui ont droit d'aller en carosse alors qu'elles y gagneraient à marcher, tandis que des personnes plus pauvres mais ayant besoin de carosse sont condamnés à l'immobilité. Il prend ensuite le cas de la famille, en prenant une idéale, sans problèmes entre les membres. Pour lui, il y a un équilibre entre les membres de la famille. Blanc prévient l'objection qu'on peut faire à son exemple, qui est que c'est l'amour qui unirait la famille et que ça ne serait pas le cas ailleurs. Donc il pren dun exemple où ça fonctionne sans amour : les clubs fréquentés par la bourgeoisie pour consommer de l'alcool et une cuisine spécifique, pratiquer des loisirs raffinés (lecture, billard, fumer dans le fumoir parce qu'à l'époque on était civilisé), le tout dans un entre-soi dont le ton est donné par la connivence sociale (imagé dans Le tour du monde en 80 jours de Verne). Blanc insiste sur la socialisation des frais. Tout le monde n'y va pas combler les mêmes besoins, mais tout le monde participe également. C'est un principe de mutualité. Il en dégage une leçon : il le décrit comme les débuts du socialisme. Ce qu'il entend par socialisme, c'est un projet en rupture avec l'inégalité bourgeoise, une société définie par la mutualité. À son époque, le socialisme englobe un ensemble de petits projets réformistes tout comme révolutionnaires, dans le but d'instaurer une société égalitaire. (On a une différence avec Marx, qui voit le socialisme comme impliquant moins de travail, et Blanc qui voit le socialisme comme avec du travail heureux.)
\par Blanc prend ensuite des cas limites pour montrer à quoi ressemblerait son socialisme, le poète et le mathématicien n'étant pas salariés facilement. Après un éloge du club anglais, il évoque d'autres formes d'une sorte de mutualisation, par les cours gratuits et autres.
\par Le texte entre dans un deuxième mouvement avec un "Mais". Blanc affirme que si un individu a le droit de profiter de ce que la société lui donne, il a le devoir d'utiliser ses aptitudes naturelles pour la société. Pour Blanc, cette utilisation est tout à fait souhaitée par les êtres humains, qui souhaitent être actifs. Ce texte s'oppose au travail en tant que torture, y voyant quelque chose de profondément bien. Les problèmes avec le travail et la paresse viennent du fait que la société soit mal organisée (hasard = hasard de la naissance ; misère). Le hasard fait que de potentiels grands artistes restent garçons de ferme (voir l'histoire de Giotto, en marge de la réalité statistique). Ce que Blanc veut dire, c'est que le hasard de la naissance est horrible, et contredit la nature. Il prend aussi l'exemple de Louis XVI, qui aurait été fait pour être mécanicien (propagande républicaine). Pour lui, c'est normal de ne pas aimer un travail pour lequel on n'est pas fait. Il convoque ensuite l'image de l'esclave noir, déplacé par le commerce transatlantique et maltraité, forcé au travail dans des champs de coton ou de canne à sucre, c'est le paradigme de l'esclavage. Pour Blanc, les esclaves sont paresseux parce que leur travail est si horrible que c'est leur seul moyen d'être heureux. C'est un contresens sur la vocation de l'homme, causée pour la société. (Un paradigme est différent d'un exemple, en cela qu'un exemple illustre de manière arbitraire et partielle tout en pouvant être substitué à un autre, là où le paradigme est le cas privilégié pour lequel on ne peut pas imaginer une illustration plus claire et plus convaincainte). Pour Blanc, le club est le paradigme d'une société harmonieuse par la mutualité. Dans un résumé, si on fait disparaître les paradigmes, le texte peut perdre en substance.
\par Il y a beaucoup de questions rhétoriques. Il mentionne dans le dernier paragraphe des exemples qui illustrent tous la même chose : on peut vouloir travailler intensément sans avoir de besoins à combler. Il finit ensuite par cet exemple cocasse d'une communauté religieuse où l'oisiveté sert de punition (retournement complet du paradigme de l'esclavage.)

\subsection{Compte-rendu des copies}
Il ne faut jamais utiliser de formules comme "l'auteur dit que". La convention est de prendre à son compte le contenu du texte. On doit respecter son système d'énonciation et son ton (même si c'est plus difficile). \par Les résumés peuvent être organiques : présenter une continuité. C'est une consigne peu respectée ; la construction en paragraphes des gens tend au morcellement, la plupart des paragraphes font une ligne et demie. Un paragraphe doit servir d'unité de sens complète, et laisser quelque chose d'isolé n'est pas acceptable. \par La synonymie est toujours en problème en résumé. On demande de résumer et pas d'étendre le texte, donc certains mots importants doivent être conservés sans synonymie. Remplacer aptitude par compétence mène par exemple à une perte de clarté dans le texte, les compétences étant acquises et les aptitudes innées. Il fallait faire apparaître l'intérêt que Blanc porte aux aptitudes que la nature donne à chacun d'entre nous, ce qui diffère des capacités de l'homme en général. Le vocabulaire du résumé doit nécessairement être précis. \par La logique est un ensemble de relation de sens qui permettent d'assembler des mots et des énoncés de manière à ce que soit ainsi produite une signification. Il y a des logiques élémentaires, comme la relation de cause à effet. Les copies sont souvent peu logiques, l'énoncé du paradigme de la paresse des esclaves a souvent mal été compris. Dans un concours, on demande d'être meilleur que les autres, pas de faire un minimum bancal. \par Quelque chose d'autre qui est mauvais dans nos résumés c'était la rigueur. La rigueur est une attitude intellectuelle qui vise à régler de la manière la plus précise un raisonnement et sa mise en forme verbale. La rigueur est impossible quand on n'utilise que des "par rapport à". La rigueur est le principal problème de nos copies. \par Deux Hommes dans la Ville est un film sur la peine de mort, parce que le prof a fait une tangente sur l'alexandrin le plus terrifiant de la langue française : tout condamné à mort a la tête tranchée. \par Le résumé n'est pas un collage d'énoncés trouvés dans le texte. Il n'y a pas de recette. Penser qu'il y a une recette mène à une pensée mécanique avec beaucoup de répétitions et peu de choses intéressantes. Qu'un connecteur logique prolifère est un grand défaut. La succession "tout d'abord, ensuite, et enfin" est à proscrire. \par Le prof a dit que le français c'était un beau langage et qu'on devait le respecter. Mes dieux ce que c'est chiant. \par Certains textes ne sont pas rigoureux, mais on doit s'en accomoder, faire un résumé qui rend compte.
\par Plusieurs remarques : certains mots sont parfois utilisées pour faire plus chic, comme "point" au lieu de "pas", qui est trop sorti d'usage pour que ça soit bien. \par L'ellipse abusive, en supprimant des mots, mène à des problèmes. Il faut se rallier à la norme linguistique. \par "Se doit de" n'existe pas, mais "doit" existe. \par Il y a trois syntaxes interrogatives : directe [inversion verbe-sujet, point d'interrogation], semi-directe [pas d'inversion verbe-sujet, mais point d'interrogation], indirecte [ni d'inversion verbe sujet, ni point d'interrogation mais une proposition interrogative avant]. \par Le "juste" français n'est pas équivalent au "just" anglais. De même, "ses envies et aptitudes" n'existe pas : il faut dire "ses envies et ses aptitudes". "Comme évoqué" est un anglicisme, il faut dire "comme nous l'avons vu". "malgré que" ne s'utilise pas non plus. \par Il ne faut pas confondre "car" et "parce que". Elles sont paresseuses parce que leur travail est sans rapport avec leurs aptitudes naturelles. Car à l'échelle d'une journée elles ne balaient que trois demi-dalles. \par Il ne faut pas confondre les mots et les maux. Les clubs ne sont pas de micro-sociétés socialistes. \par Les formes conjuguées sont plus présentes que les formes en -ant. Beaucoup d'énoncés sont clarifiables en passant à une conjugaison plus adaptées. Par exemple : le travail sera en raccord avec ses aptitudes, rendant le travail plus agréable. Il faudrait écrire : le travail dès lors sera en raccord avec les aptitudes de chacun, ce qui rendra la contribution de tous beaucoup plus agréables. \par Il ne faut pas confondre inactif et paresseux. Inactif réfère à la catégorie sociale des chômeurs, c'est une catégories objective. Les paresseux sont une catégories subjectives : des gens à qui on accorde la propriété de ne pas faire tout ce qu'ils pourraient. \par Commencer une phrase par "alors que" ou "ce qui" ne fonctionne pas, puisque ça demande que la proposition qui soit reliée par ce connecteur soit dans la même phrase : ce sont des connecteurs qui introduisent des propositions subordonnées relatives de leurs antécédents. C'est comme mettre un "et" en début de phrase. Mais après une virgule, ça va visiblement, malgré ce que disent les mamans. Le point-virgule est un canard-lapin de Wittgenstein. Il fonctionne comme un point, car il coupe la chaîne du sens et qu'oralement, la voix tombe. La seule différence c'est que la pause du point est plus petite. \par Dans un résumé, on doit conserver l'ordre du texte. On peut la transformer légèrement et ponctuellement, mais pas souvent. \par "chaque" et "tout" sont des mots différents et ne s'utilisent pas dans des syntaxes négatives. "quelque" est souvent confondu avec "quelques" et les gens les confondent. \par Dans un résumé, on doit ne pas garder les exemples, et donc ne pas écrire "par exemple".
\par Il faut rester explicite dans le résumé, on doit développer tout énoncé. \par "Il n'est pas surprenant que le repos prenne la forme de la plus grande liberté" a été repris par le professeur d'un élève. Ce n'est pas ce que dit Louis Blanc. On devrait dire : "Il n'est pas surprenant que la plus grande liberté prenne pour eux la forme de la paresse." \par truc sur la concentration que j'ai pas pris pour des raisons évidentes.
\par Le professeur a ensuite distribué un corrigé. Sur son corrigé, il nous a dit qu'il aurait dû mettre "ait le droit" en début de deuxième paragraphe. Il faudrait aussi mettre un point après "goûts" et ajouter un "Il". \par Le prof note C pour le charabia, sxe pour les problèmes de syntaxe, MD mal dit, MLD maladroit, INC incorrect, danger signifie que ça ne va pas, les chiffres sont pour les énormités, NPQI est pour "n'importe quoi", VLDT déformer le texte, LDT loin du texte, TLDT très loin du texte, OBS obscène.

\section{Nouveau sujet d'entraînement : résumé}
Alain Supiot, \textit{La Justice au travail}, 2022
\par La sécession des élites réfère à une perception selon laquelle les élites gouvernementales et politiques ne seraient plus mobilisées que par leurs propres sort et leur propre intérêt et ne se soucieraient plus du reste de la population ou ne le percevraient qu'à partir de leurs propres catégories. C'est vu comme une trahison de la part des personnels politiques dans une société démocratique. Le terme d'élite réfère à un groupe qui dans une société joue un rôle d'encardement ou de transmission ou de prescription de valeurs et de buts dans la société en question. Détruire les élites laisse un vide, voir la guerre civile algérienne dans les années 90 et le massacre de Katine (massacre des officiers polonais par l'URSS, les officiers étant tués d'une balle dans la tête par un revolver allemand). C'est une acception technique. Aujourd'hui, il a pris un sens péjoratif. \par Symétriquement à cette sécession, les gens ordinaires répondent à cette situation en se retirant du jeu politique : voter pour des gens extérieurs à la politique (le RN si on croit à sa propagande, Coluche) ou l'abstention (la crise des sociétés démocratiques). Le monde politique tourne en vase clos entre des gens venant des mêmes horizons sociaux, qui partagent les mêmes codes, le même bagage intellectuel (voir l'ENA). Au même moment, les classes laborieux se détournent du travail à cause de la gouvernance par les nombres : nos études montrent que vous devriez travailler sur ça pour augmenter votre rendement... Si les statistiques ont leur intérêt, la manie de l'évaluation venant du privé est peu attractif pour les travailleurs. Le rendement décorrélé de l'expérience de la tâche est de ne pas pouvoir souffler, ne pas pouvoir concevoir ce qu'on fait. \par L'attitude morale classique de travailler sans volonté de profit est incompatible avec le capitalisme. Le rendement déshumanise le travail. Le capitalisme conçoit le travail d'une manière incompatible avec les réserves anthropologiques (la conception du travail développées dans les millénaires précédents). Les démissions sont des symptômes de cette sécession, la crise du travail engendrée par cette obsession des nombres du gouvernement. Les travailleurs ordinaires sont mis dans l'incapacité de contribuer au bien-être commun en utilisant leurs capacités.  


\chapter{Sur La Condition Ouvrière}
Les autres oeuvres ne sont pas exigibles au premier devoir. C'est risqué de les utiliser si on ne les maîtrise pas. Le reste du cours (florilège+Weil) sera plus recommandé et totalement utilisable au cours de la dissertation.
\par L'intérêt de l'ouvrage de Weil c'est la situation, la condition qu'elle a vécu. Weil est une intellectuelle, qui vient de la bourgeoisie parisienne. Elle est cultivée, normalienne et agrégée de philosophie, comme très peu de femmes à son époque. Et elle décide de faire l'expérience de l'usine dans le cadre d'une logique propre qu'on verra plus tard. Ce qui est intéressant est la rencontre entre l'expérience de terrain et son bagage intellectuel. Elle dit que beaucoup d'analystes n'ont pas l'expérience du terrain nécessaire pour comprendre la condition ouvrière. Dans le passé, on a eu des enquêtes (comme celle d'Engels sur la misère ouvrière à Lille qui a montré l'horreur de la condition ouvrière à la France en général). Mais Weil, elle, pratique l'enquête participative. C'est un type d'enquête proche de celles des ethnologues, qui s'installent dans des villages et enquêtent sur une société (c'est ce que faisait Philippe d'Escola par exemple, avec les Lances du Crépuscule). Simone Weil anticipe le concept des "établis" (années 60-70) : des intellectuels qui, par solidarité pour la classe ouvrière, et/ou pour structurer une action syndicale, entraient dans des milieux ouvriers.
\par La caractéristique des analyses de Weil réside dans le fait qu'elle s'efforce de penser la cohérence qui caractérise à ses yeux l'usine, une cohérence qui associe l'ouvrier, la machine et la direction d'usine tayloriste. L'usine peut être définie par une unité de production d'envergure dédiée à la production en série, qui utilise des machines. Une usine est soumise à un certain type d'organisation et de direction par ses contraintes de production. On parle de taylorisme en référence à quelqu'un qui dans les premières années du XXe a réfléchi à comment organiser l'usine. Taylor a tenté d'être ingénieur mais avait des problèmes de vue, et est donc devenu ouvrier avant de devenir un encadrant. Au prix du ravalement des ouvriers à l'état d'instruments, il a créé des méthodes pour augmenter l'efficacité des usines.
\par Weil, du fait de sa méthode, va pouvoir élaborer, en orientant son expérience non pas sur le mode du témoignage individuel orienté vers des éprouvés subjects mais sur celui d'une analyse générale, va proposer une analyse de ce type d'organisations d'usine tayloriste qui a marqué l'histoire des pays développés. La satire du taylorisme, c'est Les Temps Modernes de Chaplin, avec sa scène du fauteuil à nourrir l'ouvrier : une entreprise a proposé au directeur de l'usine un fauteuil qui permettrait aux ouvriers d'avoir moins de temps pour manger. Charlot en est la victime, et la machine se dérègle, de manière drôle mais aussi terrifiante. La ligne de fuite du taylorisme, c'est la mécanisation de toutes les activités, jusqu'au fait de se nourrir. On doit contraster cette scène avec celle de la chaîne de production, où Charlot doit visser des deux mains pour garder la cadence, se retrouve à faire un travail si inhumain qu'il ne pense plus qu'à visser mécaniquement, jusqu'à se jeter sur un policier pour visser les boutons de sa veste. Cette scène montre bien la déshumanisation du taylorisme.
\par La condition ouvrière est un ouvrage posthume, qui n'a pas été conçu par Weil. C'est Camus qui a composé le recueil. Le recueil est consacré à l'analyse de la condition ouvrière, comme l'indique le titre, et prolongée par des propositions qui concernent la possibilité d'un travail non-servile (servile renvoie à l'esclavage étymologiquement, Weil utilise le mot pour insister sur le caractère décisif d'envisager d'accéder à une tâche qui ne peut pas être confondue avec l'esclavage pour les ouvriers).
\par Pour le cadre, quelques éléments biographiques : Weil, en 34, est professeure au Puy-En-Velay, parce qu'elle a voulu se rapprocher du monde ouvrier. En effet au Puy (proche de St Etienne, ville marquée par l'importance de la métallurgie et la mine, où il y a des usines et des mouvements syndicaux dont l'activité se concentre dans la bourse du travail), il y a une activité syndicale importante. Dans les villes, il y avait des bourses du travail, des lieux dédiés aux activités syndicales. Weil a enseigné la philosophie à des ouvriers en plus de son travail dans le lycée, et va manifester sa solidarité au mouvement de manière concrète en versant une grande partie de son salaire à une caisse de solidarité (pratique syndicale remontant au XIXe, qui sert à constituer un fond qui permet de tenir pendant les grèves). Toujours en 34, elle obtient un congé pour écrire une commande : \underline{Réflexions sur les causes de la liberté et de l'oppression sociale}. Ce titre annonce ses préoccupations, qui ne sont pas rares à l'époque. La misère ouvrière est très intense, un ouvrier des années 30 peut ne pas manger à sa faim, et il y a une réflexion sur les mécanismes de l'oppression sociale et sur la manière d'y mettre fin, dans le prolongement des discours des socialistes et des communistes du XIXe siècle. Ceux-ci essaient de penser une transformation conséquente de l'ordre social. Sans faire l'histoire des mouvements socialistes, un texte très marquant de cette tradition est le Manifeste du Parti Communiste en 1848, au milieu du printemps des peuples, en même temps que la chute de la monarchie de Louis-Philippe. Le texte s'ouvre sur "Un spectre hante l'Europe, le communisme" et se ferme par "Prolétaires de tous les pays, unissez-vous !", qui veut construire un internationalisme qui a pour volonté de sauver le genre humain. Dans les sociétés chrétiennes, il y a déjà un sauveur pour les âmes, et le communisme veut sauver les conditions de vie de cette partie du monde qui est de plus en plus grande. Pour comprendre Weil, il faut avoir en tête ce corpus de textes.
\par Dans ce premier livre, Weil donne une formule importante : "Le progrès technique semble avoir fait faillite puisqu'au lieu du bien-être, il n'a apporté aux masses que la misère physique et morale." Pour comprendre cette formule, il faut la remettre en contexte, en repartant du XVIIIe siècle et de l'espérance que le progrès porte dans le siècle des Lumières. Le progrès, c'est la libération des capacités de l'homme et l'exploitation de celles-ci en vue d'un bien-être individuel et collectif. Cette libération des capacités doit procéder doit procéder de la liberté. La liberté est d'abord la liberté de pensée, l'élimination du carcan que représente la pensée religieuse imposée par les églises (les penseurs des Lumières s'opposent surtout aux institutions). Il s'agit selon la formule de Kant que l'humanité sorte de son état de minorité (au sens de l'âge) pour qu'elle développe tout son potentiel. Le progrès technique fait partie du lot pour eux, permettant aux hommes d'améliorer son sort. Quand Diderot et al publie l'encyclopédie, leur but est de fournir de quoi penser, mais aussi un savoir-faire technique. À partir de là se développe un projet de transformation de la société grâce à une convergence du progrès intellectuel, social et technique. Quand Moulinex, dans les années 60, sort dans un salon une publicité exploitée très longtemps, qui disait "Moulinex libère la femme" ; en effet la vie des femmes va être transformée par ce qu'on appelle aujourd'hui la domotique : un fabriquant d'électroménager reprend la mythologie d'un objet émancipateur. Est-ce que la domotique a pour autant changé la condition des femmes ? Non, la répartition des tâches quotidiennes est encore inégale, et elles pratiquent encore la double-journée/supportent une charge mentale plus importante que les hommes. Bien que l'électroménager permette de garder du temps, comme de nouvelles tâches apparaissent notamment puisque l'exigence d'éducation est plus importante pour les enfants. C'est la problématique du genre, déterminé par la société (à bien différencier du sexe qui est biologique). Le progrès social n'a pas été à la hauteur du progrès technique, et Moulinex n'a pas libéré la femme, car la libération n'est pas un processus technique. Voir le roman d'Aragon, \underline{Cloche de Bal}. La technique a fait disparaître le poinçonneur du métro sans que l'aliénation des ouvriers de disparaisse. Ce que montre Weil, c'est que l'utilisation de machines dans des exigences tayloristes ne libère pas l'ouvrier mais sont un moyen de l'asservir.
\par La technique est neutre, elle n'est que ce qu'une société en fait. Tout pays a besoin d'une armée, Si vis pace para bellum. On n'est pas près de voir les armées disparaître. Poutine aime bien ses armées. Pour comprendre la formule de Weil, il faut bien avoir conscience qu'unne grande espérance est attachée au progrès technique. Dans la mesure où elle n'a pas été prise en charge par une société/un état, cette espérance a fait faillite. Dans les projets communistes, il est question non pas de faire disparaître les moyens de production, mais de changer la manière dont ils sont utilisés. Les 3-8 (production par équipes réparties sur trois cycles de 8H) constituent une absurdité, motivée par la rentabilité. Pour pousser à consommer autant qu'on produit, il est nécessaire que la publicité existe : si on a une paire de chaussette qui n'est plus à la mode, la pub doit nous le faire savoir. C'est le cycle fou de la production et de la consommation. L'exemple le plus flagrant de la non-libération est le travail posté : là où les individus sont biologiquement poussés à vivre le jour, ceux-ci sont obligés de travailler la nuit. Dans le recueil \underline{Aventure} ou \underline{Les amours difficiles} si on traduit mieux, on voit l'histoire d'un couple qui travaillent à heures différentes, qui ne se voient que deux moments dans la journée, le soir et le matin. Il y a une image aussi humoristique que désolante, avec des individus qui n'ont pas la possibilité de vivre une vie amoureuse normale. Tous ceux qui évoquent la nécessité de faire travailler le dimanche pour alimenter la machine économique passent sous silence le fait que la journée de repos est la dernière journée de liberté, consacrée au repos. Le progrès technique, ça n'a pas marché, et il va falloir changer de points de vue.
\par Weil parle des masses (mot qu'on n'utilise plus) pour rappeler le scandale au carré de l'inégalité : d'une part une inégalité entre très riches et très pauvres et d'autre part le fait que les pauvres constituent la majorité de la population. Ce scandale peut être traduit économiquement avec le ruissellement et le gâteau. Le ruissellement explique que les riches sont nécessaires pour consommer du luxe, et font donc travailler toute la chaîne d'individus. La théorie du gâteau dit que les biens disponibles dans une société sont comme un gâteau : s'il y a des individus avec une énorme part, d'autres en auront une minuscule (cf Obélix et son gâteau dans Cléopâtre). La théorie du ruissellement est de droite. La théorie du gâteau est située plus à gauche. L'exploitation est un mot qu'on n'utilise plus (à méditer...), définie par Marx (point de repère incontournable dans les années 30) comme reliée à la plus-value : l'entrepreneur ne rémunère pas l'ouvrier à la hauteur des richesses que son travail produit. L'entrepreneur est en situation de force dans une société munie d'un état capitaliste et peut se permettre de donner un salaire seulement suffisant pour renouveler sa force de travail à l'ouvrier. La différence entre ce que l'ouvrier produit et ce qu'on lui reverse, c'est la plus-value. Une partie des réformistes proposent de substituer l'association à l'entreprise : une production organisée de manière à ce que l'entrepreneur puisse valoriser son capita et avoir du profit mais à ce que l'ouvrier ne soit pas exploité au mépris des droits élémentaires d'une personne humaine et au prix de sa personne physique et morale. Un troisième modèle est celui du révolutionnaire marxiste (ici simplifié) : la seule solution, c'est une révolution qui supprimera la propriété privée des moyens de production et qui instaurera une société émancipée de la logique du profit (produire en fonction des besoins et pas de l'envie d'accroître sa richesse).
\par Weil est consciente de la misère matérielle des ouvriers : elle dit être inquiète d'être malade et de ne pas pouvoir réussir à atteindre ses objectifs et de ne plus pouvoir se nourrir. Un ouvrier peut souffrir de la faim tout en travaillant (le salaire aux pièces peut mener à la faim si l'on n'a pas assez produit). Cette situation est justifiée par un discours entreprenarial qui dit que si on paie trop les ouvriers, l'entreprise coulera. Encore aujourd'hui, de grands groupes réussissent à obtenir des accords d'entreprises : ils négocient, en demandant le choix au ouvriers entre le risque de la fin de l'entreprise ou de travailler deux heures par semaine gratuitement. Weil se distingue de Marx : si elle reconnaît qu'il décrit bien les mécanismes du capitalisme, il s'est trompé en faisant de la question de la question de la propriété des moyens des moyens de production la question décisive pour ceux qui veulent penser la condition ouvrière. Simone Weil en fournit ce qui est à ses yeux une preuve en prenant l'exemple de ce qu'il se passe dans la Russie soviétique. Elle critique le fait que le système de l'usine ait été maintenu par les soviétiques : Lénine disait "Le communisme/la modernité c'est l'électricité plus les soviets." Les soviétiques étaient technophiles et modernistes, l'usine est pour eux irremplaçable. Ils ont simplement remplacés les directeurs d'usines par des camarades. Elle commente que les "grrrrands" chefs soviétiques n'ont jamais les pieds dans une usine, ne comprenant pas la misère du monde ouvrier, qui est moins l'exploitation que la subordination.
\par La subordination de l'ouvrier est double : subordination aux machines et à un pouvoir humiliant, déshumanisant, exercé par la direction de l'usine et par ses relais, les contremaîtres. C'est pour elle ce facteur qui est déterminant. L'analyse de la propriété est l'analyse de l'exploitation, donc une analyse économique est financière. Mais Weil considère plus importante cette soumission qui fait perdre à l'ouvrier son humanité. Les bolchéviques n'ont donc pas mis fin au malheur ouvrier parce que leur analyse n'est pas judicieuse. Weil ne veut pas être comme eux, elle veut aller voir ce qu'il se passe, et va découvrir le caractère inhumain (au sens littéral) du travail ouvrier. Elle va s'efforcer de penser toutes les dimensions de l'oppression que subissent les ouvriers dans leur travail. Parallèlement, elle va tenter d'imaginer les conditions d'un travail qui ne serait plus servile. Quand elle insiste sur la dyssymétrie entre le directeur d'usine et les ouvriers, ce qui l'intéresse n'est pas au niveau économique (différence de salaire) mais la dyssymétrie du point de vue du pouvoir. Pour les ouvriers, "les souffrances sont inscrites dans l'essence même du travail".
\par Dans les analyses de l'époque, on a une différence frontale entre ceux qui mettent l'accent sur les exigences de la concurrence internationale et de la production, et ceux qui le font sur la nécessité d'améliorer à tous égards la condition ouvrière et qui font l'impasse sur les contraintes qui pèsent sur la production. Weil va tenter de dépasser cet antagonisme, en affirmant que la marge de manoeuvre des entrepreneurs en matière d'augmentation salariale est limité. Elle veut aussi concilier les revendications matérielles (il faut augmenter les salaires, par le biais de la disparition du salaire aux pièces par exemple) avec les revendications morales (celles qui concernent la dignité des ouvriers, ouvriers à qui on devrait permettre d'entretenir un autre rapport à la machine, un rapport qui ne serait plus de la soumission).
\par Il y a trois motifs qui déterminent l'expérience de Weil. Ce sont des motifs psychopolitiques. Les ouvriers ne peuvent pas penser le malheur qui les accable, il faut donc que quelqu'un analyse pour eux leur situation et essaie d'envisager les mesures qui permettraient d'échapper à ce malheur. Weil refuse le paternalisme, donc aller travailler en usine est une manière d'éviter ce reproche : elle parle à partir de l'expérience commune des ouvriers, qu'elle a connu elle-même. Elle va étayer ses analyses sur un certain nombre de connaissances intellectuelles. Citons quelques exemples d'expériences participatives : Dans la peau d'un Turc (journaliste allemand sur le traitement et l'exploitation des travailleurs immigrés en Allemage, ce qui a fait scandale en montrant qu'on demandait aux turcs de faire des choses dangereuses comme nettoyer des cuves toxiques, avant de les payer au noir), \underline{Le Quai de Ouistréham} (d'une journaliste qui a été enlevée par un groupe terroriste au moyen-orient avant d'être relâchée et de faire ce livre plus tard, et elle raconte les choses invraisemblables de mépris et d'exploitation qu'elle y a vécu, son nom est Florence Aubenat).
\par On peut mentionner d'abord des motifs politiques et pratiques : Weil veut associer la théorie et l'expérience pour avoir plus de chances de penser ce malheur ouvrier. Il y a en dernier lieu des motifs personnels, dans le sens où il y a une notion qui joue un rôle essentiel dans la conscience Weil, et c'est la compassion. C'est ce que Rousseau appelle la pitié, la capacité à s'identifier au malheur d'autrui. La compassion joue un rôle essentiel dans la perspective chrétienne. Parlons de la religion de Weil : elle est d'une famille juive déjudaïsée, et elle va s'intéresser au christianisme certes sans se convertir, mais va adhérer d'une part moralement et d'autre part à partir d'expériences mystiques (après 37), après quoi elle a la conviction d'être entrée en contact avec le Christ. Dans une lettre à Thévenon, elle dit qu'elle s'identifie à une prostituée et à tous les autres opprimés, c'est un exemple de compassion. Elle veut faire l'expérience de ce malheur, non pas par masochisme, mais par cette compassion qui la mène à s'identifier au malheur et ainsi trouver des moyens de trouver des solutions.
\par Le cours sera construit en trois grands moments, centrés sur les trois pôles de la démarche de Weil :

\section{Les raisons d'un choix, celui de l'expérience du travail en usine.}
\subsection{Le silence des ouvriers}
Les ouvriers ne commentent pas leur condition, c'est le constat dont part Weil. Elle dispose d'une thèse pour l'expliquer : pour que les ouvriers puissent parler, donc penser et critiquer leur sort, il faudrait qu'ils disposent d'une capacité de réflexion face à leur condition et qu'ils soient en mesure de faire acte de résistance. Or l'oppresion qu'ils subissent rend toute résistance durable impossible. Ils ne sont donc pas en mesure de parler de leur condition, et ceci d'autant moins que cette condition se définit par le malheur. Ce que Weil appelle le malheur, c'est quelque chose de précis : la situation concrète et psychologique qui étouffe l'esprit. Quand le malheur disparaît pendant quelques heures, la conscience s'efforce de l'oublier au lieu d'essayer de le penser ; "Les ouvriers peuvent très difficilement écrire, parler ou même réfléchir [à leur malheur] car le premier effet du malheur est que la pensée veut s'évader ; elle ne veut pas considérer le malheur qui la blesse." Ce malheur est celui de la subordination. Les acteurs humains de la subordination sont le directeur d'usine qui dicte les conditions de vie en usine, et aussi le contremaître, qui est socialement proche des ouvriers mais qui est du côté du patron, qui est là pour faire respecter les consignes organisationnelles du directeur.
\par Le contremaître est une figure honnie, un traître, membre de ce que Weil appelle la "police de l'atelier". Il faut savoir qu'en 34, la direction n'a pas l'obligation d'autoriser les syndicats. Aujourd'hui, à partir d'une certaine taille, tout entrepreneur a l'obligation de procéder à des élections syndicales. Ce n'est pas le cas à l'époque, et donc protester contre la manière dont se passent les choses c'est risquer a minima l'humiliation d'une réprimande publique et a maxima le renvoi. Le patron n'a pas besoin de justifier ses renvois, ce qui a changé aujourd'hui, le droit du travail encadrant ça (mais pas très bien). Mais en même temps, aujourd'hui les CDI sont de plus en plus rares, et la sécurité de l'emploi du statut de fonctionnaire devient de plus en plus enviable. Même dans les milieux universitaires, on tend vers la situation anglo-saxonne, où tous les postes d'universités sont maintenus par des projets de recherche.
\par Selon Weil, "l'humiliation a toujours pour effet de créer des zones interdites dans lesquelles la pensée ne s'aventure pas". Ces zones sont recouvertes par le silence ou par la mauvaise foi. Cette situation, qui est le fruit d'une oppression très lourde, explique pour Weil que les ouvriers ne pensent pas leur condition et n'en parlent pas. Ce n'est ainsi pas une question d'indigence de la part des ouvriers. Weil pousse les ouvriers à prendre la parole, pour ne pas se retrouver dans la situation paternaliste où un individu "éclairé" parle à la place de tous les autres. Le paternalisme a de forts liens avec le totalitarisme. Weil imagine des dispositifs matériels pour résoudre cette situation, comme l'installation de boîtes à suggestions : boîtes où les ouvriers anonymisés peuvent déposer ce qu'ils ont consigné par écrit, concerant leurs propositions et doléances sans prendre de risques. On revient à la question de la parole dans l'usine. Dans ses lettres à Victor Bernard, Weil dit "J'ai cru comprendre qu'à l'usine, il est interdit de causer sous peine d'amende." Deux choses sont à prendre en compte. Il s'agit d'abord d'une atteinte à une liberté fondamentale de l'être humain et une conception du travail qui débouche sur une situation qui, du fait de la soumission au primat du rendement et de la rentabilité va à l'encontre d'un fait humain élémentaire, qui est que les humains aiment parler. "Il y a mille moyens de faire parler une femme et pas un pour le faire taire" et "Bats ta femme tous les matins, si tu ne sais pas pourquoi, elle le sait." sont des proverbes qui illustrent le fait que parler est naturel pour l'humain.
\par Il est impossible de conserver la parole et le travail sur la presse. Weil, au contraire, pousse les ouvriers à prendre la parole, et en attente de cette parole, elle éprouve la nécessité d'élaborer un discours sur l'expérience des ouvriers mêmes, donc de faire expérience de la condition ouvrière et de cette oppression.

\subsection{Théorie et expérience dans la condition ouvrière}
Weil est persuadée que la souffrance ne peut être pensée que si on la met en relation avec ses causes effectives et aussi seulement quand on l'a éprouvée. La question de la souffrance dans la condition ouvrière est une question décisive dans l'absolu (voir \textit{Germinal}) et pour Weil, puisqu'elle est mue par la compassion. La compassion est déclenchée immédiatement par la souffrance. Elle commente le choc qu'elle a subi pendant son expérience ouvrière : "Cette expérience qui correspond par bien des côtés à ce que j'en attendais, en diffère quand même par un abîme : c'est la réalité, non plus l'imagination. Elle a changé toute ma perspective sur les choses, le sentiment même que j'ai de la vie." Cette formule justifie à ses yeux l'expérience. Elle dit ici qu'elle n'a pas été surprise à proprement parler : on sait déjà des choses sur la vie en usine, mais entre le savoir et l'expérience il y a une différence, et la confrontation à cette différence est pour elle non seulement une épreuve mais aussi une sorte de révélation puisque c'est toute sa relation à l'existence qui est changée. Dans la suite de la citation, elle parle aussi de la joie, disant que sa joie (qui a un rapport avec la pensée chrétienne de Weil) sera altérée par son expérience. Elle découvre la multiplicité des formes concrètes que prend la misère des ouvriers. Il faut consigner cette multiplicité, et que pour cela il est nécessaire de citer des anecdotes, de colporter des choses qui peuvent sembler des détails mais qui en réalité permettent de comprendre la condition ouvrière et son malheur.
\par On trouve chez Weil une préoccupation de la douleur ouvrière, le refus d'une démarche trop intellectualisée et le désir d'informer plus que de témoigner. La prise de connaissance des détails dont elle parle va lui permettre d'élaborer des analyses qui vont au-delà de son expérience individuelle et qui ont une dimension politique (qu'on verra plus tard dans une anecdote significative) puisqu'ils permettent de décrire un certain type de relations de pouvoir et ce qu'un groupe social ffait à un autre dans une société donnée (société française des années 30). Weil décrit une situation où il n'y a pas de contre-pouvoir, or ce qui est définitionnel de la démocratie (sous forme de régime comme de pratique sociale), c'est l'existence de contre-pouvoirs. La tyrannie est caractérisée par leur absence (pas de recours face à un abus : exemple limite quoique parlant du policier qui a le pouvoir de répression mais qui peut être mis en cause par un simple citoyen, bien que le fonctionnement de ce contre-pouvoir puisse être questionné ; on a ici un exemple de séparation des pouvoirs judiciaires et exécutifs dans une démocratie). Sans contre-pouvoir syndical, les individus sont en proie à l'arbitraire et à l'oppression.

\subsection{Portraits de Simone Weil}
L'entreprise et la manière d'écrire dessus par Weil est révélatrice de tendances de Weil. Dans la deuxième lettre à Albertine Thévenon, Weil dit "Je me sens la soeur de la fille qui fait le trottoir - de tous les êtres méprisés, humiliés, maniés comme du rebut." Weil prend l'image de la prostituée, l'individu objet du mépris social par excellence. Il y a une résonnance chrétiente : le Christ est celui qui acceuille parmi ses disciple tout le monde, même les méprises (comme Marie-Madeleine). Il n'y a pas de réprouvés, seulement des victimes d'une situation de domination d'après Weil. L'intérêt de Weil pour tous les dominés est indissociable du désir de changer la condition de ces individus, et ce changement devrait être radical : elle dit que "Il ne suffit pas de vouloir leur éviter des souffrances, il faudrait vouloir leur joie." Cette formule est importante, définissant deux crans : si un certain nombre de personnes peuvent souhaiter que les dominés ne souffrent plus sans aller plus loin, ce n'est pas assez, il faut \textbf{vouloir} que tout un chacun accède à la joie. Par vouloir, elle dit qu'il faut que cette exigence se présente à la conscience. Ce n'est pas une question de moyens. Pour revenir sur son rapport à la chrétienté, elle vient de la bourgeoisie parisienne juive mais déjudaïsée, sinon peut-être une ou deux fêtes par ans. Elle découvre la spiritualité chrétienne, et sans se convertir formellement, elle sera très marquée.
\par On retrouve cet autoportrait dans la démarche d'identification de la condition ouvrière. Elle va s'adapter à la condition ouvrière, s'efforcer de se comporter comme une ouvrière malgré une santé défaillante, ce qui lui posera un certain nombre de problèmes. Elle prend un engagement significatif : elle dit "Je me suis jurée de ne pas sortir de cette condition d'ouvrière avant d'avoir appris à la supporter [au point que je puisse] y conserver intact le sentiment de ma dignité d'être humain." A contrario, ça signifie qu'en devenant ouvrier, on laisse derrière sa dignité humaine. C'est une attitude faite de résistance que veut développer Weil. Elle avoue qu'elle n'est pas encore parvenue à cela : sa dignité d'être humain a été mise en pièces comme l'est celle des ouvriers. Elle souhaite rendre une sorte d'hommage au monde ouvrier en reconquérant (elle qui est immergée dans cette expérience mais qui est armée pour l'analyser) cette dignité.

\section{Le travail ouvrier, une condition oppressive et déshumanisante}
\subsection{Les déterminants techniques et humains du travail des ouvriers}
Dans les années 30, ces caractéristiques découlent à la fois des particularités des machines et des techniques de production en série. La grande caractéristique est l'organisation du travail en tâches partielles et spécialisées, celles du travail à la chaîne. L'autre caractéristique est propre à la gestion humaine du travail, qui est mise en oeuvre par les directeurs d'usines, relayée par les contremaîtres. Pour faire apparaître ces caractéristiques, Weil compare rapidement la situation de l'ouvrier d'usine et celle de l'artisan, comparaison commuen. Ce qui caractérise le travail de l'artisan, c'est qu'il manie ses outils selon un rytme qui lui est propre, qu'il déploie un savoir-faire et qu'il peut développer son habileté dans la réalisation de gestes variés. À l'inverse, l'ouvrier est soumis à la cadence de la machine (l'opposition entre le rythme et la cadance est très importante). L'artisan évolue dans un espace qu'il s'est naturellement approprié, celui de son atelier. Son espace est organisé en fonction du travail sans exclure des formes d'agrément. Dans l'usine, l'espace est entièrement soumis aux exigences de rendement (voir Les Temps Modernes, qui caricature le rendement dans la machine à faire manger l'ouvrier, qui sous prétexte de rendement torture les ouvriers).
\par Un bon exemple d'organisation de l'espace dans le règne de l'entreprise est le concept d'open spaces, où chaque bureau est visible depuis les autres, de manière à ce qu'on puisse mener plus efficacement les projets avec du travail en équipe. Ces open spaces impliquent une promiscuité : la sphère d'intimité des individus  disparaît pour satisfaire des exigences de rendement. Ces open spaces illustrent aussi la manière dont des institutions/entreprises peuvent tenter de prendre en compte comment se sentent les ouvriers pour le rendement. C'est le cas de ces entreprises qui encouragent à venir avec leurs animaux de compagnie, ce qui n'est pas fait que pour le bien-être des employés mais aussi pour qu'ils travaillent mieux. Ce qu'on appelle la chaîne de production est, d'une manière ou d'une autre, un aménagement de l'espace et du rapport à l'espace qui permet d'optimiser le geste de l'exécutant. Pour mesurer les implications de ce type d'organisation, il faut bien comprendre que l'ouvrier n'est plus ici qu'un moyen de mener à bien le processus de production. Kant disait "Il faut toujours autrui non pas comme un moyen mais comme une fin." Dans le cadre du travail en usine, c'est le contraire qui se produit. Cette situation met en cause la dignité de l'individu.
\par Les machines comme la machine à boutons d'Alstom, la presse et la visseuse ont toutes une puissance de contrainte qui force l'ouvrier à se plier à des mouvements qui sont déterminés pas les exigences de la mécanique et qui heurtent la logique naturelle des mouvements du corps. Ceci implique évidemment de la fatigue et des pathologies professionnelles parce que le corps travaille dans des conditions et dans des postures qui sont peu compatibles avec les exigences du corps. Si aujourd'hui l'Etat se préoccupe des maladies professionnelles, ce n'est pas que pour leur prix cher pour la société mais ça relève de la responsabilité de notre état social de prendre en compte le travail. Un exemple limite est celui du cas de l'amiante, souvent utilisé dans l'isolation, bien qu'on ait su assez tôt que ça provoquait des cancers. On a laissé les rapports dormir pendant longtemps jusqu'à ce que le scandale explose, et les industriels ont été obligés de reverser de l'argent aux familles des gens empoisonnés à l'amiante. Dans \textit{Dans la peau d'un Turc}, l'auteur journaliste raconte les travaux de nettoyage qu'on peut donner aux immigrés turcs, payés au noir et toxiques.
\par Le travail parcellaire (à la chaîne) se caractérise par le fait que les employés répètent toujours le même geste. Plus la tâche est mécanisée, plus le travailleur est efficace. Cette organisation présente plusieurs inconvénients : en premier l'ouvrier n'est pas associé à l'ensemble du processus de production d'un objet, alors son travail perd son sens (il est mécanisé pour accomplir une tâche et pas formé pour une activité), deuxièmement cette parcellisation a amené la rémunération à la tâche (travail aux pièces, qui n'existe plus en France aujourd'hui) qui mène à une relation obsessionnelle à l'argent et au temps. Weil raconte sa peur de ne pas réussir à atteindre la vitesse de production minimale, et d'être renvoyée. On aperçoit l'image d'un monde où la vitesse seule est valorisée et la lenteur humaine est éliminée. "Il faut bien que tout le monde vive" reste vrai, et cette obsession de la production est un problème délicat, puisqu'il faut définir une limite. Ce que dénonce Weil, c'est d'une part cette dimension dangereuse (en prenant le cas limite du cercle vicieux de l'ouvrier trop lent, qui repart avec un salaire qui ne permet pas de bien se nourrir, ce qui fait baisser son rendement et ainsi de suite jusqu'à la mort par épuisement) mais aussi le fait que dans l'usine, le travail obéit à "une organisation purement bnureaucratique" (première lettre à Albertine Thévenon), qui ne prend pas en compte le facteur humain. Cette organisation produit un monde dans lequel les ouvriers sont isolés les uns des autres et sont réduits à l'état de molécules (dit dans l'article "La rationalisation", texte le plus substantiel du recueil).
\par Weil voudrait que l'usine soit un lieu de vie, et quand elle évoque l'atmosphère des grèves de 36, elle met l'accent sur la joie collective et l'harmonie entre individus qui s'installe. "Ce n'est pas le travailleur qui utilise la condition de travail mais la condition de travail qui utilise le travailleur."
\par Il est nécessaire de mentionner Taylor, à l'origine de la rationalisation dans les usines et du traitement des ouvriers. Taylor a vécu entre 1856 et 1915, et a commencé très tôt à réfléchir à l'organisation des usines. Il naît en 56, ne peut faire d'études d'ingénieur à cause d'un problème de vue pour devenir directeur d'usine quelques années après avoir commencé à travailler. Dans \textit{La Direction Scientifique des Entreprises}, 1911, Taylor propose trois mesures : la promotion du salaire aux pièces, l'utilisation de normes de chronométrage et la mise en place d'une "police des ateliers". Ces trois mesures sont des clés importantes pour comprendre tous les sujets de dissertation. Le travail aux pièces implique selon Weil une soumission morale et un épuisement physique, à cause des normes de chronométrage imposées en dépit du bon sens. Les normes de chronométrages empêchent l'élasticité du temps alors déshumanisé, et correspondent à une obsession de la mesure (proche de l'obsession de l'évaluation actuelle). La police des ateliers induit une pression permanente, impose aux ouvriers des interdits. Taylor disait que cette police avait pour but d'empêcher la lutte des classes, en forçant les ouvriers d'entrer en compétition et en leur interdisant de se concerter.
\par Sur la lutte des classes : une classe est un ensemble de social définie sur des critères dont les revenus, la position dans la hiérarchie sociale et la culture propre à ce groupe social. Selon Marx, les sociétés modernes sont caractérisées par une exacerbation des conflits entre classe, où tout ce que gagne une classe est perdu par l'autre. L'horizon de la lutte des classes dans la société industrielle selon Marx, est qu'après que les prolétaires se soient découvert une conscience de classe, ils se rendent compte qu'ils vivent dans une condition de dominés déterminée par des règles du jeu social qui profitent à une classe de la société et défavorisent la leur. Pour Marx, cette prise de conscience mènera à des luttes ouvrières qui pourront déboucher sur une révolution qui construirait un nouveau monde, défini par la justice et l'égalité. On trouve chez Marx beaucoup de considérations dialectiques, héritées d'Hegel : c'est dans la situation où les ouvriers seront le plus malheureux qu'ils pourront se rebeller. Le maximum d'oppression mène alors à la plus grande liberté. En prenant en compte la baisse tendancielle du taux de profit, les ouvriers auront un salaire si mauvais qu'ils seront prêts à faire la révolution, mettre fin à une société caractérisée par l'inégalité. Quant à savoir si le socialisme réel a dévoyé le projet de Marx ou si c'était la seule chose qui pouvait en sortir. Pour Marx, que les employés se rendent compte qu'ils ont plus en commun avec leurs concurrents qu'avec leurs patrons permettra une société plus juste.
\par Simone Weil insiste sur la récurrence des menaces de renvoi dans l'usine. Cette menace est permanente, et permet de maintenir la subordination. Une des dimensions injuste de cette menace est que l'entrepreneur n'a pas besoin de justifier ces licenciements. Aujourd'hui, le droit du travail est plus encadré, afin de limiter les licenciements abusifs. Weil, à l'inverse, parle d'une forme de terreur mise en place dans le lieu de travail, par le biais du renvoi. Le statut de fonctionnaire, qui est censé être un exemple pour les organismes privés de comment l'emploi devrait fonctionner, est aujourd'hui précaire et repose sur l'avancement à l'ancienneté dans un monde où il est difficile d'être viré. La sécurité de l'emploi, acquis si important, est un problème quand les gens ne font pas leur travail. Cette stratégie de terreur que décrit Weil est organisée comme une espèce de spectacle. La spectacularisation du châtiment, de l'infamie est quelque chose qu'on retrouve dans les pays totalitaires. C'est quelque chose de choquant, d'inadmissible. Les renvois ont des conséquences dramatiques en termes de subordination et d'avilissement des ouvriers. On peut mourir de faim dans le monde de l'usine (repenser au cercle vicieux du mauvais travail qui mène à ne pas assez manger et obliger à faire du travail plus mauvais encore jusqu'à l'inanition).
\par Lettre à Simone Gilbert (SG) : "On vit à l'usine dans une subordination perpétuelle et humiliante, toujours aux ordres des chefs." nous dit Weil sur l'avilissement des ouvriers. Le management contemporain s'efforce de dissimuler l'autorité, au point de ne plus parler de chefs. C'est une bonne chose, mais les relations mêmes n'ont pas changé.
\par Le désir de montrer que les ouvriers ne comptent pas (ils sont remplaçables) est permanent et la puissance du directeur est "une puissance de Dieu plutôt que d'homme" (Dans la même lettre). Simone Weil le montre en parlant des ouvrières qui attendent sous la pluie l'heure de rentrer dans l'usine alors que les chefs ont une entrée réservée qu'ils prennent quand ils veulent. On ne se soucie pas de la santé des ouvriers (on voit aussi cela dans \textit{Un vrai crime d'Amour} où on voit les gens attendre d'entrer dans l'usine, exposés à la météo).
\par La justification de ces conditions strictes est l'impératif du rendement. Pour comprendre l'importance de ce phénomène, Weil mentionne la concurrence. Le capitalisme industriel est un phénomène international dont l'existence est solidaire de celle du libéralisme économique (à différencier du politique). Le libéralisme se traduit par une disparition des droits de douane, ce qui crée une situation de concurrence internationale. Le patron peut utiliser l'argument de la concurrence pour justifier des salaires modestes (euphémisme), en disant que ces salaires permettent l'obtention d'un produit à bas prix, qui se vendra donc mieux. Symétriquement, le patron peut brandir la menace de fermer l'usine s'il ne baisse pas les salaires. Il est donc obligatoire d'accepter les réductions de salaires, l'alternative étant de ne pas avoir de salaire si l'entreprise ferme. "Une automobile ne sert pas seulement à rouler sur une route, elle est aussi une arme dans la guerre permanente que mènent entre elles la production française et celle des autres pays." Le primat du rendement fait que "les choses jouent le rôle des hommes, les hommes jouent le rôle des choses ; c'est la racine du mal." (Expériences de la vie d'usine) nous dit Weil. Derrière cette formule, on retrouve la définition de l'humanisme, où l'homme est la mesure de toutes choses. Pour reprendre les termes de Kant : l'injonction première de la morale, c'est qu'il ne faut jamais traiter autrui comme un moyen mais comme une fin. Les hommes deviennent des moyens de produire dans l'usine, alors qu'ils sont la fin : dans une société, tout ce qui est produit devrait avoir pour seule fin l'harmonie individuelle et sociale. C'est une critique radicale du capitalisme industriel, la \textbf{racine} du mal
\par La notion d'expérience renvoie ici à l'immersion dans une situation qui va permettre d'élaborer une analyse critique pour donner une prise intellecutelle sur cette situation.

\subsection{Les corps ouvriers au travail}
La dimension physique dans la condition ouvrière s'inscrit dans le corps des travailleurs. Elle insiste sur le fait que l'ouvrier vit une misère qui est à la fois morale et physique. (Morale définie par opposition à physique, pas seulement en tant que règles du Bien et du Mal.) Pour comprendre l'imbrication de la souffrance physique et morale, on peut reprendre l'analyse de la discipline par Michel Foucault. Un exemple qu'il prenait pour illustrer la discipline est celle de l'évolution des armées au XVIIIe : le bon soldat n'est plus celui qui maîtrise les domaines militaires et qui apporte quelque chose d'important, le bon soldat c'est celui qui a intériorisé par l'exercice des techniques du corps, ce qui fait de lui un rouage dans une machine et à se comporter comme telle (ne pas avoir de réaction psychologique, et obéir aux ordres). La discipline, c'est un ensemble de techniques de dressage qui vont se généraliser pendant le XVIIIe et le XIXe, par exemple à l'école. Les idées disciplinaires estiment qu'une bonne posture est relié à un bon comportement en tant que membre d'une communauté.
\par Les corps des ouvriers connaissent une forme particulière de fatigue, de faim et de blessures. Commençons par le plus simple, avec les blessures. Dans un monde industriel sans beaucoup de normes de sécurités, les blessures se multiplient : quand une machine demande à un ouvrier de présenter une pièce à une cadence régulière, il y a des risques que l'ouvrier ne respecte pas la cadence et se retrouve avec une main sous une presse. Les normes de sécurité qu'on voit aujourd'hui n'existaient pas à l'époque, puisque ça coûte de l'argent, rendant la machine plus chère. Imposer des protocoles fait de plus chuter le rendement, donc les entrepreneurs tâchent d'éviter ces difficultés. L'ouvrier, s'il est fatigué, va commettre plus d'erreurs. Et un corps fragilisé par la fatigue subira des blessures plus graves.
\par Or, l'ouvrier est dans un état de fatigue permanent d'après Weil. Pour aborder les choses de manière plus synthétique, Weil dit que les ouvriers vivent sous le signe de la souffrance : "C'est une souffrance physique très longue ou très fréquente. Une telle souffrance est souvent tout autre chose qu'une souffrance. C'est souvent un malheur." Elle insiste sur le mot de souffrance, l'utilisant pour décrire un état dominant, une sorte d'altération de la condition humaine. Par là, Weil revient à la question de la subordination : "L'étau de la subordination leur est rendu sensible à travers les sens, le corps." Elle ajoute que le pire n'est pas la durée du travail, mais l'intensité. Le résultat de cette vie marquée par la subordination et la souffrance est "l'écoeurement". Un écoeurement qui ne caractérise pas seulement le rapport au travail, mais aussi le rapport à soi, à ce que l'on est et à sa propre existence. Le rapport à soi est donc dominé chez l'ouvrier par un malaise dont l'individu ne peut pas se libérer.
\par Au-delà de son analyse de la condition ouvrière, Weil réfléchit aux conditions d'instauration d'un travail humain et conforme aux exigences de l'humanisme, elle développe cette partie dans son dernier article : Conditions premières d'un travail non servile. Concrètement, cette condition ouvrière dominée par la souffrance, la faim et les blessures peut conduire l'ouvrier à la mort dans le cadre du cercle vicieux évoqué plus tôt, où le faible rendement mène à la sous-alimentation et à la diminution du rendement jusqu'à la mort. Cette analyse et ce raisonnement ont une qualité heuristique : ils aident à interpréter la condition ouvrière et le malheur qu'il définit. Simone Weil explique dans une lettre qu'elle en est venue à être "obsédée" par la perspective d'un bon repas. Comment est-il possible de travailler convenablement en étant obsédé par la faim ?
\par Weil évoque la souffrance physique structurelle qu'elle a éprouvé, par exemple en travaillant dans les fours chez Alsthom. Ces fours fonctionnaient pour fondre du métal, nécessitant de grandes chaleurs. Elle a travaillé près des fours pendant un moment : "Un chef, en me contraignant à risquer deux heures durant de me faire assommer par un balancier, m'a fait sentir pour la première fois pour combien au juste je comptais : à savoir, zéro." (Lettre à Victor Bernard) Pourquoi compte-t-elle pour zéro ? Parce qu'on considère que son intégrité physique est moins importante que la machine, et que si elle est mise hors d'état de travailler, on peut la remplacer (les ouvriers sont interchangeables). Aujourd'hui, on considère qu'une mort qui pourrait être évitée est sans prix (voir les drames entre chasseurs et promeneurs). La dignité humaine individuelle des ouvriers est ignorée par la hiérarchie.

\section{Comment penser la possibilité d'un travail non servile}
Le point de départ de Simone Weil, c'est l'analyse de la condition ouvrière sous le régime du machinisme (dans le cadre du capitalisme industriel). Dans le prolongement de son analyse critique, elle tente de définir ce que seraient les caractéristiques d'un travail qui respecterait non seulement les besoins physiques du corps humain mais aussi les besoins spirituels de l'individu. Il faut préciser ici que dans cet ensemble de textes, Weil alterne entre des approches matérialistes (l'usine étant une structure conçue pour satisfaire les exigenres du capitalisme industriel, elle est dominée par l'exigence de rendement, productiviste comme financier, qui détermine la mise en place de cadences infernales qui conduisent à une condition ouvrière destructrice du corps et de l'intégrité psychologique et morale de l'individu) et spiritualistes (l'homme, créature spirituelle, doit être nourri spirituellement). C'est par cette analyse spirituelle que Weil arrive à dire dans \textit{Condition Première d'un travail non-servile} qu'il faut que l'ouvrier soit en rapport abec la beauté et avec Dieu.
\par Elle rappelle qu'il est absolument nécessaire de bannir la souffrance de la journée de travail, et que ça implique que l'ouvrier jouisse de conditions de rémunération, d'alimentation, de logement "convenables" et ajoute même qu'il faut que l'ouvrier bénéficie de loisirs (puritaine, elle ne cnsidère pas comme un loisir l'alcool et la débauche). Comparé à ces idées, son assertion suivante, dire qu'on va sauver les ouvriers en orientant leur conscience vers Dieu pose deux problèmes. D'abord, nos réflexes intellectuels font que ce n'est pas par là qu'on commencerait une réflexion, mais aussi qu'en tant qu'héritiers des Lumières, nous considérons illégitime de prétendre détenir une vérité exclusive, particulièrement une vérité religieuse (les Lumières se sont construits contre l'institution religieuse du catholoicisme qui prétendait détenir la vérité exclusive). Cette idée de Weil nous paraît donc contraire à la tolérance, la possession d'une vérité exclusive étant pour nous une prétention menant au fanatisme. Ce type de démarches ne fait plus partie de notre cadre de pensée. Dans son contexte, en 1942, l'espérance révolutionnaire est forte, et ne fait pas de place pour Dieu, les communistes accusant l'église d'avoir été complice de l'oprression ("La religion c'est l'opium du peuple" de Marx, une simplification extrême des pensées de Marx à l'gard de la religion).
\par On peut être surpris en passant des analyse matérialistes de Weil à ces perspectives religieuses surtout développées dans le dernier article. Si elle insiste tout au long du receuil sur la sorte de dégradation morale subie par les ouvriers du fait de leur condition, et qu'elle dit à demi-mot que les ouvriers n'ont que peu accès à la dimension spirituelle de l'amour. Les premiers textes du recueil datent de 1934, les derniers de 1942, et elle vit un cheminement philosophique qui la mène à déplacer l'accent de son analyse vers sa dimension spirituelle. Dans son livre de 1932, \textit{Réflexion sur les causes de la liberté et de l'oppression sociale}, il y a deux parties, d'abord Analyse de l'oppression et ensuite Tableau théorique d'une société libre. Elle affirme dans ce dernier article être à la recherche d'une norme en fonction de laquelle "tous les problèmes de la technique et de l'économie doivent être formulés". Selon elle, on ne doit pas céder à un argumentaire qui se référerait uniquement aux exigences techniques et économiques, qui ne doivent pas déclasser les exigences morales, sociales ou spirituelles.

\subsection{Le caractère technique et matériel d'un travail non-servile}
Lorsque Simone Weil essaie de penser la transformation de la condition ouvrière, elle prend en compte bien sûr la question des machines et en l'occurrence, elle réclame la conception de machines qui permettraient l'expression et la satisfaction de deux besoins humains élémentaires et universels :\begin{itemize}
\item Le besoin de prendre des initiatives
\item Le besoin d'éprouver la fierté devant le travail accompli.
\end{itemize}
Aujourd'hui, on a en partie dépassé le taylorisme, le management contemporain laisse une place à l'initiative individuelle, et on s'efforce de donner la possibilité d'être fier de son travail ce qui passe par une culture d'entreprise. Si le nouveau management a pris ce virage, c'est parce qu'on a constaté qu'en offrant au travailleur des conditions qui lui permettent de s'investir d'avantage dans son travail, les résultats étaient meilleurs. Ce n'est pas un processus philanthrope, mais une volonté de productivité. \par On parle beaucoup aujourd'hui de "l'ajonctivité". Dans les sciences sociales, l'ajonctivité est ce qui fait en sorte que les individus aient une possibilité de mettre en oeuvre leur énergie/inventivité. L'ajonctivité, c'est faire en sorte que l'individu soit dans les condition pour mobiliser ses capacités à agir. C'est l'idée derrière les micro-projets (financer très peu des projets très petits et tangibles) et pas les grands projets d'entraide internationale. Dans tous les cas, on peut tracer une ligne droite entre ce que dit Weil et les pratiques contemporaines qui relèvent de l'ajonctivité.
\par Selon Weil, l'autre besoin est celui d'être fier de son travail. Être marchand d'armes permet de gagner énormément d'argent, mais ce n'est pas nécessairement gratifiant de le faire, pour en donner un exemple. Ce que Weil veut promouvoir un véritable progrès, qui ne "pèse pas sur les ouvriersé et ne constitue pas une exploitation plus grande de leur force de travail. Dans l'article sur la rationalisation, elle expliquait que dans l'usine, le progrès était technique et menait à une exploitation plus importante de l'ouvrier. Weil, dit que "la transformation des machines peut seule empêcher le temps des ouvriers de ressembler au temps des horloges". Le temps des horloges réfre à la mesure arithmétique très précise, par des chronomètres, alors que le temps humain est dominé par la notion de ryhtme. Ce qui est terrible dans le machinisme cpour Weil, c'est que l'homme est au service de la machine et pas le contraire. Elle explique qu'il faudrait, pour transformer les machines, remplacer les machines semi-automatiques soit par des machines complètement automatiques, soit par des machines "souples" (qui impliquent une plus grande intervention de l'homme, avec des capacités de réglage considérables, au point que l'essentiel du temps de l'ouvrier y serait consacré, plus polyvalentes donc, tout en permettant aux ouvriers d'être plus spécialisés). Elle ne donne cependant pas d'exemples de machines. Dans la chaîne de production, il existe déjà des régleurs qui savent régler des machines et les réparer en partie.
\par Il faut que l'usine cesse d'être un lieu d'oppression et que ses conditions de vie ("la nourriture, le logement, le repos et les loisirs doivent être tels qu'une journée de travail soit normalement vide de souffrance physique") soient meilleures. Cela impliquerait que l'état le garantisse. Il faut donc selon Weil un état social, réglementariste. Elle écrit en 1942. En 1936, il y a eu les accords de Matignon, qui amènent la semaine de 40H et la semaine de congés payés. Weil est très en phase avec le sens de l'histoire dans les pays développés, mais que l'application laisse encore à désirer. Il faut des inspecteurs du travail et des syndicats pour que ces conditions soient vérifiés. Les accords de Matignon ont imposé l'élection de délégués du personnel, pour créer un contre-pouvoir face aux actionnaires. Le SMIG (Salaire minimal interprofessionnel garanti) n'arrivera qu'en 1950. C'est le CNR, conseil national de la résistance, qui a élaboré ces réformes.
\par Insistons sur les congés payés. On a tous l'image des jeunes couples qui se ruent en tandem vers les plages pour la première fois de leurs vies. La mer, c'est tout un imaginaire touristique à l'époque, puisque la plupart des ouvriers n'ont jamais vu la mer de leurs vies. Symboliquement, c'est très fort, c'est l'accès à autre chose pour tout le monde. Aujourd'hui, un français sur deux n'a pas les moyens d'aller en vacances.
\par Dans l'article Condition d'un travail non-servile, Weil évoque la semaine de 30H, affirmant que certaines revendications salariales sont trop grandes pour des entreprises et que la diminution du temps de travail serait plus facile. Elle dit que "Des milliers d'ouvriers pourraient enfin respirer, jouir du soleil, se mouvoir au rythme de la respiration, faire d'autres gestes que ceux imposés par des ordres". On trouve l'expression d'une protestation contre un monde de l'usine qui fait de l'ouvrier une créature mécanique. "Jouir du soleil" est un moyen de rappeler que l'homme doit entretenir un lien avec le cosmos, ce qui fait d'autant plus sens à l'époque où se développe la mode du bronzage. Dans ses expressions, on revoit la question du rapport au temps, de l'opposition entre un rythme humain et une cadence machinique. Elle rappelle que "l'homme a soif d'un sentiment de propriété du temps".
\par Dans le prolongement de cette analyse, Weil affirme qu'il faut en finir avec le salaire aux pièces. En effet, le salaire aux pièces rend l'individu obésdé par le gain, il devient soumis à la machine et ne reçoit qu'une rémunération très faible.

\subsection{Les caractères moraux qui seraient ceux du travail non-servile}
Il faudrait que l'arbitraire des réprimandes et ders renvois disparaisse. Plus largement, il faudrait que les relations entre les ouvriers et la direction changent. Dans un article précédent, elle proposait cette boîte anonyme pour que les ouvriers posent leurs doléances. Mais en 1942, elle écrit que "la règle dans les échanges de vue [entre les encadrements et les ouvriers] devrait être une égalité totale entre les interlocuteurs, une franchise et une clarté complète".
\par Certains part du principe que patrons et intérêt ont fondamentalement des intérêts antagonistes, faisant de l'usine un modèle réduit de la lutte des classes (qui selon Marx permet d'expliquer toutes les tensions de la société pour être simpliste). Soit on se rallie à ce modèle, soit on prend le modèle de la collaboration, qui est convaincu qu'un terrain d'entente existe. En France, le syndicalisme est historiquement dans le premier modèle, tandis qu'en Allemagne on a un meilleur exemple du modèle de la collaboration. Weil, réticente quant à la révolution, mue par la compassion et un désir d'unanimisme, voudrait plutôt valoriser la collaboration. C'est ce qui la mène à correspondre avec Victor Bernard, même si ça ne fonctionnera pas bien (Victor Bernard a peut-être été tué par des résistants, lié à un parti d'extrême-droite et anti-syndicaliste). Quand Weil évoque la transformation de la condition ouvrière, elle ne milite pas pour la disparition de l'effort. Ce qui lui pose un problème n'est pas que les ouvriers doivent faire des efforts mais que ceux-ci soient déshumanisants et sans gratifications.
\par Elle oppose à cela le paradigme du travail agricole. Avant de travailler en usine, elle a expérimenté le travail dans les champs et affirme que malgré la difficulté des efforts (dus en partie à sa santé fragile), elle avait sentie une gratification. Moissonner, préparer le blé pour l'alimentation, préparer la paille pour le bétail était satisfaisant, ça justifiait le travail accompli. Mais le résultat du travail en usine ne justifie pas la souffrance imposée aux ouvriers. 
\par Ce qu'elle réclame dans l'usine, c'est la possibilité de réaliser une forme de fraternité. "Un sourire, une parole de bonté, un instant de contact humain est une grande valeur" dans l'usine. Elle a fait l'expérience du contraire de la fraternité, un monde où les ouvriers étaient en concurrence les uns contre les autres, où le contact humain n'a pas cours. Elle a fait l'expérience des grèves de 36, une appropriation de l'usine par les ouvriers, et une transformation totale. Avec ces fêtes, elle a vu la possibilité d'un autre monde.
\par On voit une certaine naïveté chez Weil dans son idée de faire visiter l'usine aux familles des ouvriers, en opposition à ces moments à attendre sous la pluie d'entrer alors que les patrons rentrent immédiatement.

\subsection{La dimension spirituelle du travail non-servile}
On entre dans une partie relativement étonnante, reliée au mysticisme. En 1937, Weil entre en contact avec le Christ. Le mysticisme recherche un contact avec la divinité, qui abolit la médiation de la raison sur la réalité, qui a une dimension hallucinatoire. C'est banal dans la culture mystique, mais surprenant pour notre culture.
\par Ce qui est important, c'est que la manière que Weil a de concevoir sa relation aux ouvriers est infléchie par ce contact mystique : "J'ai eu soudain la certitude que le christianisme était par excellence la religion des esclaves, que des esclaves ne peuvent pas ne pas y adhérer et moi parmi les autres" (dans une lettre à un ami religieux) Il est vrai que le christianisme s'est d'emblée présentée comme la religion des pauvres, et s'est opposée à l'origine aux autres religions qui soutenaient les institutions et aux cultes qui se présentaient juste aux dominés. Le christianisme se considère universel, avec une prédilection pour les pauvres ("il est aussi difficile pour un riche d'arriver au paradis que pour un chameau de passer par le chas d'une aiguille", "les premiers seront les derniers"), ceux qui souffrent et que le christianisme va consoler. Cette notion passe au premier plan pour Simone Weil, surtout après ses expériences mystiques.
\par Ce n'est pas parce qu'on est dominé qu'on ne jouit pas de sa dignité d'être spirituel. Ce que Weil affirme dans le dernier article de son recueil, c'est que l'essentiel est de restituter cette dignité spirituelle aux ouvriers.
\par "Il n'y a pas le choix des remèdes. Il n'y en a qu'un seul, une seule chose rend supportable la monotonie, c'est une lumière d'éterntité, c'est la beauté." "La vocation de l'homme est d'atteindre la joie pure à travers la souffrance." Quand Weil parle de vocation, on devrait parler d'une disposition de l'homme qui rend possible le fait d'atteindre la joie pure en allant au-delà de la souffrance, en lui donnant un sens métaphysique particulier. Ces deux phrases introduisent une dimension surprenante pour nous. Il se produit chez Weil une évolution d'une analyse matérialiste de la condition ouvrière à une interprétation spiritualiste. Une interprétation matérialiste est celle qu'elle donne dans les premiers articles : c'est l'analyse du taylorisme, qui part de la manière dont se joue le rapport entre l'homme et la machine (l'homme est au service de la machine, cette soumission devient une exploitation dont l'intensité est indéfinie et illimitée). Le machinisme tayloriste s'inscrit dans le cadre du capitalisme industriel, dominé par la loi du profit. Cette analyse matérialiste permet de dégager des conséquences physiques mais aussi morales et psychologiques (les ouvriers sont conduits à la lâcheté en ne cherchant que le profit, perdent leur dignité et ont une relation "dominée par l'écourement" avec les autres comme avec eux-mêmes). Cette analyse matérialiste n'est pas surprenante pour nous. En revanche, lorsque Weil opère un saut vers une analyse spiritualiste, on est dépaysés. Dans cette analyse, la dimension matérielle passe au second plan et laisse sa place à la dimension spirituelle. C'est ce qu'il se passe quand elle dit que le seul moyend e sauver les ouvriers est la beauté.
\par Appeler la beauté une lumière s'inscrit dans les perspectives chrétiennes de Weil. La lumière est une composante symbolique du christianisme, métaphysique et spirituelle. Ici, parler de beauté renvoie à Dieu. La joie pure, c'est l'état d'esprit de celui qui vit dans la proximité de Dieu, ce qui change son rapport à l'existence, à soi-même et aux autres. Qu'est-ce qu'une expérience mystique ? Dans un dictionnaire de théologie, c'est la "rencontre intérieure et unifiante de l'homme avec l'infini divin" (ici, unifiante signifie que l'individu a l'impression de se fondre en Dieu, en cet infini divin au-delà de la raison, donc on ne peut le percevoir que par l'intuition ou l'éblouissement [on revoit la métaphore de la lumière]). Pour avoir une expérience mystique, il faut deux choses : d'abord pour le sujet il faut que la raison soit désactivée (voir "Credo quia absurdum" de tertunien, qui dit que le mystère de Dieu est plus important que la raison, le mysticisme est une radicalisation de cette conviction, et développe de manière rationnelle malgré des prémisses rationnelles des idées très précises, dénoncées par Voltaire dans la lettre anglaise dédiée à Pascal, qui considère que la raison doit s'atteler à la question du comment, tandis que Pascal mobilise sa raison pour la théologie, la question du pourquoi), et ensuite il est nécessaire que l'objet contemplé soit "bien au-delà du sensible".
\par On peut partir de la notion d'"attention intuitive", contrastée avec l'attention requise par le Taylorisme qui est inhumaine (dictée par la cadence de la machine) et déshumanisante (car conduisant à l'abrutissement, absorbant l'esprit dans quelque chose sans noblesse). L'attention intuitive pourrait être proche de ce qu'on appelle la méditation aujourd'hui : se concentrer sur quelque chose qui est de l'ordre de la transcendance, de la vie de l'esprit, des réalités supérieures (Dieu, la place de l'homme dans l'univers...). Elle appelle cet état intuitif parce que tout homme aurait pour elle (et c'est une idée pascalienne) une intuition de leur rapport à Dieu, qu'on peut canaliser pour trouver Dieu et trouver la joie. Dans le cadre d'une dialectique très surprenante, Weil dit que les ouvriers sont particulièrement bien armés pour la recherche de Dieu parce que leur tête a été évidée de toutes les connaissances, leur misère intellectuelle étant un avantage pour accéder à cette intuition divine.
\par On a commencé à lire l'extrait en page 426, au dernier paragraphe. C'est un extrait qui fait une analogie (Aristote définit l'analogie comme le raisonnement de la forme "A est à B ce que C est à D") entre la personne qui travaille avec des poids et ldes leviers et le Christ sur sa croix. Cette analogie fonctionne sur un principe physique, avec l'idée que si le Christ fait un petit acte, avec la distance infinie entre le ciel et la terre, permet un grand changement. Au lieu d'être un accablement (le ciel est si loin de la terre), la distance est en fait un moyen d'action de Dieu (Weil opère un retournement dialectique). Ce raisonnement apparaît par l'intuition, qui porte sur la possibilité d'une interprétation théologico-mystique de la condition ouvrière. Simone Weil va donc expliquer que pour changer sa situation, il faut chercher dans la réalité des indices analogiques qui permettent pendant l'attention intuitive de retrouver la beauté divine. L'indice analogique, c'est que les ouvriers soulèvent des poids et que le christ a soulevé les péchés du monde.
\par Le paragraphe suivant est proche d'un sermon, mettant en relation les concepts théologiques et des que peuvent vivre les fidèles. Elle élabore un raisonnement qui est de l'ordre du sermon, afin d'obtenir une conversion du regard. Elle parle, scientifiquement, de la photosynthèse. Quand elle dit que la lumière du soleil meut nos moteurs, c'est parce qu'il s'agit de déchets organiques comme le pétrole, qui viennent tous d'une forme de la chlorophylle. Elle mentionne la pesanteur, idée très présente dans sa pensée et opposée à la grâce. Quand elle parle de capter la chlorophylle, elle parle de l'agriculture, la première chose que firent Adam et Eve pour aider l'humanité à se multiplier. Et le Christ est associé à la cholorophylle, parce qu'il nourrit ses fidèles avec du pain et du vin (ceci est ma chair, ceci est mon sang, versé pour vous en rémission de vos péchés), aussi bien physiquement que spirituellement.
\par Dans le paragraphe suivant, elle relie la mécanique à la théologie, et fait de son travail un moyen de méditer sur sa condition. Elle dit en somme que l'individu peut trouver partout des moyens d'analogie dans l'attention intuitive, et généralise en parlant des comptables et des caissiers. Quelques paragraphes plus tard, elle parle de la poésie, qui doit être surnaturelle : faisant le lien entre l'homme et Dieu. En disant que c'est quelque chose qui était vrai au moyen-âge, elle fait une erreur historique. Et surtout, elle dit des choses qui sont assez proches de ce que disait Pétain ("La terre, elle, ne ment pas"), bien qu'elle y soit très opposée idéologiquement.
\par Sur l'inégalité, elle dit que la poésie pourrait permettre une jonction entre les ouvriers et les intellectuels, sans aucune inégalité. Il y aurait une "plénitude de l'attention" : cette attention mystique au Christ deviendrait une prière. Des travaux séparés dans leur spécialisation mèneraient tout de même à une égalité devant la prière. De là, elle en vient à considérer que l'inégalité sociale est moins importante que l'inégalité spirituelle qu donne à chacun sa dignité d'être humain. Et au praragraphe suivant, elle affirme que l'école devrait enseigner aux enfants à trouver l'attention intuitive. Elle affirme que si tout le monde comprenait que tous les travaux pouvaient mener à la plénitude céleste, alors l'égalité réelle apparaîtrait. Un marxiste, s'il entendait ça, sortirait sans doute une arme contre son interlocuteur. "La représentation tout à fait précise de la destination surnaturelle de chaque fonction sociale fournit sseule une norme à la volonté de réforme." Chaque fonction sociale offre un moyen d'accéder à l'attention intuitive. La volonté de réforme doit donc s'appliquer à permettre à chaque fonction sociale d'atteindre l'attention intuitive. La souffrance doit donc être bannie de la condition ouvrière.
\par Nous avons ensuite traité le paragraphe de la page 432 commençant par "Une certaine subordination". Elle affirme que l'uniformité et la souffrance ont des mesures : il en est des quantités qui sont ménageables et qui ne font pas de mal, mais qu'au-delà de ça l'ouvrier est dégradé et que le responsable est coupable d'un crime. Il y a aussi l'affirmation que le travail est toujours lié à une souffrance qu'on ne peut pas supprimer, ce alors que le discours progressiste se construit sur l'idée fondamentale que le travail sans souffrance peut exister. Elle se contredit un peu plus bas sur ce point. Elle ne dit pas où elle met la frontière de la nécessité de la souffrance. Ce texte a été écrit vite, et secontredit. Elle reprend l'idée du désir du superflu, et considère que toute publicité/propagande devrait être considéré comme un \textbf{crime}. Elle affirme aussi que si on ne peut pas se passer de l'autorité, on peut se passer de l'arbitraire. Pour elle, la solution est d'usiner la pièce dans de petits ateliers d'artisans.
\par Dans le paragraphe commençant par "Mais le pire attentat", elle reprend l'idée de crime. \textbf{"La basse espèce d'attention exigée par le travail taylorisé nest compatible avec aucune autre, parce qu'elle vide l'âme de tout ce qui n'est pas le souci de la vitesse. Ce genre de travail ne peut pas être transfiguré, il faut le supprimer."} Cet extrait est à savoir synthétiser parfaitement. Dedans, elle affirme que le travail taylorisé ne peut pas être sauvé, car il ne peut pas être transfiguré (il ne peut pas être intégré dans les tâches avec une vocation spirituelle). La souffrance et la dégradation infligée par le travail taylorisé ne sont pas supportables, il faut donc le supprimer. Dans un monde dominé par le machinisme, c'est une proposition révolutionnaire.
\par Ensuite, elle affirme que tous les problèmes économiques et sociaux peuvent être conceptualisés en prenant en compte le sort des travailleurs. Mais encore une fois, elle ne définit pas de séparations ni de limites. Dans le dernier paragraphe, elle dit "Il ne suffit pas de vouloir leur éviter des souffrances, il faudrait vouloir leur joie." Par joie, elle ne veut pas dire la débauche, trop puritaine pour ça.













\section{Analyse d'extrait (pages 317-319)}
Après avoir parlé de Taylor et de ses innovations productives, elle mentionne Ford et son idée de chaîne de production avant de commencer l'extrait.
\par Le dernier paragraphe est très ironique : aux yeux de Simone Weil, on ne pourrait prétendre créer l'harmonie sociale au prix de l'intégrité morale et physique des individus. Soit Taylor est un crétin, qui ne comprend pas l'harmonie, soit c'est un cynique.
\par Pour comprendre les enjeux de la discussion, il faut comprendre son arrière-plan philosophique et son ancrage dans la pensée de Kant. Du fait de cet ancrage, elle refuse de conssidérer la raison comme la seule "raison instrumentale" (ce qui permet d'organiser les choses). Chez Kant, la raison et la science mène à une meilleure connaissance de la nature et de l'homme, ce n'est pas que la technique. Là où Taylor parle de rationalisation, Weil dit qu'il faudrait parler de contrôle. Weil se sert de sa formation philosophique pour délégitimer cette organisation scientifique du travail. Weil appelle plus tôt la chaîne de travail comme "une méthode perfectionnée pour \underline{extraire} des travailleurs le maximum de travail dans un temps déterminé." Avec le verbe "extraire", Weil établit que le capitalisme de Taylor considère que l'ouvrier n'est qu'une source d'énergie, ce qui fait surgir une vision très radicale de l'exploitation capitaliste. Chez Marx, l'exploitation c'est la plusvalue (l'écart entre la rémunération de l'ouvrier et la richesse produite). Chez Weil, l'exploitation c'est une stratégie de confiscation de l'énergie vitale de l'individu. Comme ce processus est illimité, il a pour seule limite la mort donc la négation de l'ouvrier comme être humain. Le capitalisme industriel est de l'ordre de la démesure au sens philosophique. Un peu avant l'extrait, elle a insisté entre l'opposition entre les ouvriers qualifiés et les manoeuvres spécialisés, à qui on demande d'exécuter une tâche. Un des torts que l'usine fait aux ouvriers, c'est la disparition de l'activité manuelle et pas mécanique.
\par Ce que Weil appelle contrôle dans le monde de l'usine, c'est la répression qui interdit de délaisser la production ou de se syndiquer, mais aussi l'univers de la mesure arithmétique de l'usine, par le chronométrage des tâches, mesuré du point de vue de la machine et de la production. C'est contre cet univers de la mesure où l'ouvrier n'est qu'une variable d'ajustement, cette fausse rationalisation qui extrait l'énergie aux ouvriers, que Weil propose un exemple. Elle pose une machine avec des limitations techniques (absence d'un acier rapide, un alliage particulièrement résistant qui est utile dans la coupe, le perçage ou le cisaillage d'aures métaux). Mais tout à coup, un acier rapide apparaît, on peut doubler la cadence. Dans l'univers tayloriste, on dirait à l'ouvrier qu'il devrait travailler deux fois plus. Mais la solution de Weil serait d'embaucher un autre ouvrier, qui ne coûterait pas plus cher puisqu'on produirait le double de pièces. L'ouvrier n'est alors plus la variable d'ajustement. Elle imagine aussi un monde où les bénéfices sont redistribués. Dans cette sitation, l'ouvrier, l'entrepreneur et le consommateur seraient gagnants (plus de pièces veut dire qu'elles sont moins chères). Cependant, elle oublie un problème qui est le coût d'investissement. Le progrès technique peut être social aussi, mais c'est un choix politique et éthique. Des logiques pourraient être rentables, mais l'éthique interdit de les développer. Surtout, l'ouvrier ne doit pas être la variable d'ajustement.
\par On arrive à une opposition entre durée et intensité. La durée peut être chiffrée, mais l'intensité du travail ne peut pas être chiffrée, du moins pas aussi facilement. Elle cite alors le paradigme du coureur de Marathon : après la victoire des grecs à Marathon contre les perses, un messager court jusqu'à Athènes pour passer le message de victoire et meurt d'épuisement en arrivant. C'est un personnage héroïque. Mais l'ouvrier exploité jusqu'à la mort, qui meurt dix jours après avoir quitté l'usine, qui perd une main à l'usine, c'est tout autant un héros, la victime d'un système qui a sombré dans la démesure du mépris de l'être humain. C'est ce que Weil convoque avec ce paradigme. Elle passe à la notion de record, la démesure par excellence. Peut-on imaginer la même chose pour l'industrie ? C'est ce que fait le Stakhanovisme de l'URSS, avec son héros qui pulvérise tous les records de productivité. Mais Weil dit que la production ne doit pas être le monde des records, où on jette les gens qui ne sont pas encore totalement usés parce qu'ils ne sont plus à leur meilleur. Ils n'auront pas droit à une retraite, mourant trop tôt et recevant une pension trop petite. C'est une démesure incompatible avec le respect de l'être huamin.


\section{Méthode de dissertation}
À propos de ce qu'on attend et qu'on n'attend pas de nous dans la dissertation :
\par Il ne faut pas considérer qu'on doit élaborer une dissertation de philosophie générale sur le travail, puisque ça nous mènerait à dire des banalités (nous ne sommes pas munis des armes philosophiques pour réfléchir à des sujets généraux) et que ce n'est pas ce qu'on attend de nous. Il faut confronter le sujet à ce qu'il se passe dans les trois oeuvres. Il est nécessaire de faire jouer les trois oeuvres les unes contre les autres. Il ne faut pas construire un argument général en prélevant des exemples dans les trois oeuvres, mais, à partir des oeuvres et du cours mémorisés, construire un plan basé sur notre connaissance du cours.
\par Puisque les trois oeuvres n'ont pas le même discours, on peut être amené dans différentes parties à commenter des textes différents. La dissertation est un exercice artificiel, considérer que ne pas mentionner une oeuvre dans une sous-partie est dramatique conduit à ignorer la logique. Il faut valoriser les similarités et les différences entre les oeuvres, ne pas se servir des oeuvres que pour illustrer son propre propos.
\par L'objectif est de faire dialoguer les trois oeuvres, donc à partir de leur texte. Donc faire une partie entière sur une oeuvre est proscrit, les comparaisons doivent avoir du sens. Ce week-end, il faudra aussi utiliser le florilège, puisque nous n'avons pas vu toutes les oeuvres. Ce qui est extérieur aux trois oeuvres dans les concours ne doit pas consister plus de 5 pourcents. Montrer sa culture est toujours valorisé, mais la dissertation doit rester centrées sur les oeuvres.
\par L'originalité dans le traitement d'un sujet sera valorisée. Le cours est trop massif pour entrer dans une seule copie, il faut donc trier les analyses les plus pertinentes, et pris les plus originales pour y gagner. Il vaut mieux rendre une copie courte mais bien écrite. Une bonne copie serait une bonne copie où le correcteur, après l'introduction, sait où la copie veut aller. Une bonne copie, c'est aussi une copie qui tire le correcteur dans un sens, qui a l'air de savoir où elle va. Une mauvaise copie n'a pas de direction, et s'arrête brusquement, sans logique.
\par Quand on a une difficulté dans l'écriture un paragraphe, il faut imaginer un public pour écouter la phrase, et donc simplifier l'expression pour sa compréhension. On peut écrire moins, mais en écrivant quelquechose qui aura forme humaine. En effet, notre langage est souvent peu clair, mais il faut tenter de le rendre meilleur.
\par Il n'y a pas de plan magique pour la dissertation. Plutôt que de vouloir opposer les textes, il faut traiter puis dépasser le sujet dans une troisième partie (mettre le doigt sur ce que les points aveugles du sujet, bien que le hors-sujet soit un risque). Pour Weil, comme elle écrit dans un contexte historique différent du notre, c'est donc à traiter dans ces parties.
\par Il ne faut pas se demander si on est d'accord ou pas avec un sujet. Le correcteur s'en fiche. Il ne s'intéresse qu'au cheminement intellectuel. C'est peu modeste que d'aborder les choses de cette manière. Il faut se demander de quelle manière ce sujet nous fait lire les oeuvres. Le bon sujet, c'est celui qui permet de bien lire les oeuvres.
\par Il faut rendre justice aux oeuvres, ne pas leur enlever leur subtilité. Les traiter comme des recueils de lieux communs bien-pensants n'a aucun intérêt. Le correcteur doit avoir l'impression que nous sommes capables de circuler dans l'oeuvre.
\par Sur l'introduction : elle est séparée en trois temps. En dernier, on explicite son plan. Avant ça, on formule la problématique, qui doit être claire et rapide. Et en premier, il n'y a rien de mécanique. Il faut introduire la copie, et ça se fait de manières très différentes. On peut par exemple faire une référence à une oeuvre littéraire, et la relier au sujet. Une introduction sans allusion au sujet est une erreur. S'il est trop long, on ne le citera pas en entier, mais on citera des morceaux. Pour l'accroche, prendre un risque ne vaut que rarement la peine.
\par Sur la conclusion, on commence par dire "nous avons vu que [], que [] et que []", puis "Nous sommes en mesure d'affirmer que []" et en dernier une ouverture. Cette dernière est risquée, puisque souvent peu claire et ne disant rien. Si l'on n'a rien à dire, alors autant ne pas parler.

\section{Plan Weil dialectique}
Le titre de l'exposé récapitulatif du professeur est : Simone Weil, pour une lecture philosophique et engagée du monde de l'usine. Sur ce titre : au lieu de parler du travail en usine, le professeur a choisi d'en parler comme d'un monde, une sorte d'enclave à part. Elle est construite à partir du monde social qui l'entoure, dominé par des formes d'inégalité et un état d'esprit capitaliste. Mais ce qu'il se passe dans l'usine est particulier. Que Simone Weil estime ne pas pouvoir en parler sans y avoir travaillé le prouve. D'autres romanciers ont élaboré des dystopies qui sont des extrapolations du monde de l'usine (voir\textit{Metropolis} de Fritz Lang ou \textit{Les Indes Noires} de Jules Verne). La dystopie est définie par opposition à l'utopie, une société présentée comme idéale et encadrée par des règles très strictes, où tous les aspects de la vie sont encadrés ; dans le \textit{Meilleur des Mondes} de Huxley, la vie amoureuse même est encadrée par l'état qui interdit des relations à long terme tout en encourageant la fécondité et en distribuant des drogues gratuitement, ce qui illustre comment ces utopies sont construites dans l'objectif de bannir le désordre. Sur la notion d'engagement, proposée par Sartre après la fin de la seconde guerre mondiale, et donc anachronique à l'oeuvre de Simone Weil. Sartre dit que les intellectuels doivent prendre parti pour comprendre la société dans laquelle ils vivent. Weil a ce projet en tête, et a payé de sa personne, bien plus que Sartre et de Beauvoir qui distribuaient juste des journaux aux sorties des usines.
\par Introduction : Il faut rappeler que depuis l'apparition du capitalisme industriel au XIXe siècle, de nombreuses analyses ont été produites sur la notion d'exploitation, qui sont partie prenante d'une réflexion plus large sur la misère ouvrière et l'injustice sociale. L'originalité du travail de Simone Weil c'est qu'elle pense la question du machinisme, c'est-à-dire d'un monde du travail qui a été pensé non pas en fonction du travailleur (qui serait la perspective d'un travail humaniste) mais en fonction du potentiel et des exigences d'une part de la machine et d'autre part des intérêts des entrepreneurs et des actionnaires. Cette analyse du machinisme, Weil la mène avec les outils philosophiques que lui a fourni sa formation (elle est normalienne et agrégée de philosophie) et aussi mue par le primat chez elle d'une éthique de la compassion. Pour mener cette nalayse, elle est forte d'une véritable expérience de la vie en usine. Ici, expérience renvoie au fait de se plonger dans une situation qui va affecter votre perception du monde, votre relation aux autres et à vous même.
\par Cette expérience du travail et de la vie en usine lui permet notamment de parler de ce qu'on peut appeler le corps ouvrier (si on dit le "corps ouvrier", on met l'accent sur le fait qu'il est façonné par la condition ouvrière). Elle peut parler des implications spirituelles, de la manière dont est traitée ce corps et plus largement cette personne. Nous allons voir les ancrages et les acquis de cette réflexion, pour comprendre son articulation.

\subsection{Une expérience menée au nom d'une éthique et en fonction d'un ancrage philosophico-politique}
L'éthique est proche de la morale, concernant la manière dont il faut vivre. L'éthique d'un individu est la manière dont il considère qu'il faut vivre d'après des valeurs.
\subsubsection{La compassion}
Dans une des lettres à Albertine, Weil dit qu'elle s'identifie à "la fille qui fait le trottoir". La prostituée est le paradigme de l'individu méprisé. Weil s'identifie donc à ceux qui sont méprisés, dominés. Elle s'identifie à la prostituée par compassion, sans avoir rien à voir avec ce monde. La compassion est une valeur chrétienne, le fait que l'individu animé par la conviction que tous les êtres humains sont ses frères (ils sont tous les enfants de Dieu) ne peut que souffrir de la souffrance d'autrui, et se préoccuper de la manière dont il pourrait remédier à cette souffrance. Cette compassion de Weil ou cette pitié chez Rousseau est un des arrières-plans de la vertu de la fraternité républicaine. C'est un arrière-plan éthique de la démarche de Simone Weil, qui insiste sur le fait qu'elle se sépare de ceux qui mènent seulement des analyses idéologiques, politiques, sociales. Pour sa part, elle met en jeu sa propre personne. C'est pour elle une forme de solidarité que d'éprouver la même souffrance qu'une grande partie de la population. Elle veut éprouver cette compassion, et protester contre la disparition dans l'usine de tout ce qui relève de l'attention portée à autrui. Cette disparition est le fruit des relations de domination qui existent entre d'une part l'entrepreneur et ses contremaîtres et d'autre part l'ouvrier. Cette attention n'existe même pas entre les ouvriers eux-mêmes. Et cela tient à la vie qu'on leur fait, à leur situation de concurrence. Même au sein de la famille, a priori dans laquelle les individus sont liés, la brutalité de l'exploitation fait qu'on arrive à des situations paradoxale, comme celle de l'épouse qui éprouve du soulagement quand l'époux meurt, après qu'il ait été incapable de travailler ou le fait d'accepter la mort d'un enfant. Dans son article sur la grève des ouvrières, elle ne condamne pas cette situation, elle se contente de la noter.
\par Simone Weil veut faire cette expérience, et découvrir un monde social qu'elle ne connaît pas. On est dans les années 30, et le monde social est bien plus hiérarchisé qu'aujourd'hui. Weil n'a pas eu d'occasions dans son existence de frayer avec des gens d'une classe inférieure. On peut voir dans la biographie de Simone de Beauvoir l'étrangeté d'une situation de séparation sociale. Sartre de même, raconte l'histoire de son passage au front, où il rencontre des gens d'un environnement bien plus modeste. À l'origine, le service militaire menait à faire se rencontrer des gens qui ne se seraient pas rencontrés sinon dans la république. Weil veut faire ce type d'expériences, et ne plus s'inscrire dans ce modèle d'enfant de la bourgeoisie qui critiquent l'usine. Elle se moque ainsi des idéologues bolchéviques qui ne sont jamais allés dans les usines et qui n'en ont pas changé le fonctionnement, gardant une espèce de mépris de classe envers les ouvriers.

\subsubsection{L'expérience VERSUS la théorie}
Simone Weil aurait pu se contenter d'une analyse rationnelle et documentée à partir de rapports, de témoignages, mais elle a voulu tenter cette expérience de la vie ouvrière. Elle se définit par opposition au modèles des idéologues bolchéviques, qui prétendent transformer la réalité sociale pour le bonheur du peuple (lire \textit{La République des Animaux} d'Orwell, satire du Marxisme-Léninisme où les cochons bolchéviques prennent le pouvoir et exploitent tous les autres animaux). Elle valorise une opposition entre l'expérience et la théorie, voulant absolument faire l'expérience de la condition d'ouvrière. Cependant, elle est armée dans son analyse de ses outils de philosophe, et mobilise ce qu'elle sait dans des articles.

\subsubsection{L'engagement sans le surplomb}
Il y a eu beaucoup de philosophes et intellectuels qui ont analysé la condition ouvrière. Weil, elle, ne veut pas "parler à la place des ouvriers". Elle montre que les ouvriers ne parlent pas de leur condition. Parler de leur condition, ce serait la dénoncer, et pour ce faire, il faudrait que les ouvriers s'émancipent du régime de terreur qui est celui de l'usine et qu'ils acceptent de poser leurs pensées sur leur propre condition. Or, ils ont bien conscience que leur condition est une forme d'avilissement, et donc ils évitent cette confrontation et au contraire cherchent à s'évader, à oublier leur condition par l'ébriété, la luxure. C'est pour exu un moyen de survivre, mais ça rejoint ce que dit Weil : les ouvriers sont dans une telle sitaution qu'ils ne peuven tpas se mobiliser, à part dans ce miracle que constituent les grèves de 36. Ce déficit de parole, Weil prétend y remédier, d'abord en proposant une boîteaux lettres dans laquelle les ouvriers pourrraient anonymement proposer leurs doléances. Elle fait l'impasse sur le fait qu'écrire est difficile pour certains ouvriers : s'ils sont alphabétisés, passer par l'écrit n'est pas facile pour des gens qui ne sont pas familiers avec. De plus, un entrepreneur qui accepterait ce genre de système aurait déjà basculé de l'autre côté, aucun entrepreneur qu'elle ne décrit ne veut prendre soin des ouvriers. On retrouve aussi un exemple de naïveté avec son exemple des tours, qui néglige le prix des machines. Weil ne veut pas endosser le rôle de l'intellectuel qui parle à la place de l'ouvrier. Dans la mesure où elle fait l'expérience de la condition ouvrière, elle parle à partir de la condition ouvrière, et donc elle n'a plus une position de surplomb paternaliste.

\subsection{Penser la subordination et le malheur}
\subsubsection{De la souffrance au "malheur" : cadence, chronométrage, terreur}
Quand Weil parle de la souffrance indéfectiblement liée à la condition ouvrière, elle parle d'une souffrance qui n'est pas ponctuelle, qui engendre un état permanent. Parce que la souffrance devient permanente, elle engendre le "malheur". Cette souffrance est faite de fatigue (une relation à soi et au monde extérieur caractérisée par la fatigue), de faim, de douleur. Tout cela tient à la question de la cadence, la vitesse de la machine qui impose à l'ouvrier une vitesse d'exécution. La cadence s'oppose au rythme, la vitesse "naturelle" du corps et de la conscience. La cadence est inhumaine, et elle manifeste l'assujession de l'imme à la machine. C'est aussi par le chronométrage que se renforce la cadence. Le chronométrage détermine la rémunération. Weil affirme que tout le système est conçu à partir d'une ignorance délibérée des besoins et des aptitudes non seulement du coprs mais plus largement de la personne humaine. Enfin, l erégime de terreur instauré par le contremâitre va de l'interdiction de parler jusqu'à la pratique du renvoi public et sans justifications. Le pouvoir de l'entrepreneur ressemble plus à celui de Dieu qu'à celui d'un homme dans ce domaine. C'est un monde d'injustice.

\subsubsection{Du corps dominé au naufrage spirituel}
Simone Weil montre que le corps est marqué par la violence qu'on lui impose et que l'esprit est lui aussi soumis à un régime de violence qui est celui d'une sorte d'abrutissement puisque toute la conscience de l'individu est en quelques sorte aspirée par le caractère répétitif de l'activité imposé par la machine. L'entrepreneur considère l'ouvrier comme une simple source d'énergie. Ce mélange fatigue, de souffrance et d'abrutissement caractérise la condition ouvrière. Simone Weil conclut à un naufrage spirituel parce que l'ouvrier ne peut plus user de ses capacités intellectuelles, ne peut plus entretenir une relation saine avec le temps (la dimension du projet existentiel et de l'espérence disparaît) : le temps n'est plus que celui de la répétition mécanique déterminé par la cadence de la machine. D'autre part, la manière dont l'individu est traité se traduit par son avilissement. C'est pour cela que Weil proteste contre les leaders ouvriers et syndicalistes qui ne se battent que pour augmenter les salaires. Il est bien sûr nécessaire d'augmenter les salaires, mais elle voit aussi des enjeux spirituels.

\subsubsection{Redéfinir l'exploitation et condamner le taylorisme}
Chez Marx, l'exploitation est interprétée essentiellement en termes économiques, c'est la plusvalue. Chez Simone Weil, ce que l'entrepreneur exploite, ce sont les ressources vitales de l'ouvrier. L'image de l'électricité peut être reprise et transformée, l'entrepreneur est une sorte de vampire qui exploite les ouvriers. C'est une menace physique pour les individus. L'entrepreneur cherche à réclamer toujours plus d'intensité chez l'ouvrier : s'il n'est pas mort au bout d'une heure de travail, c'est qu'on peut lui demander une intensité encore plus grande. Tout cela est le fruit du taylorisme, qui a imposé le chronométrage, le travail aux pièces et la police des ateliers. La forme la plus avancée du taylorisme est le montage à la chaîne de chez Ford. Le taylorisme est caractérisé par une obsession de la mesure et du contrôle.

\subsection{Une critique radicale}
\subsubsection{La déshumanisation du travail et la trahison de l'humanisme}
Le travail humaniste permet à l'individu doit l'émanciper, contribuer à l'harmonie sociale. À l'inverse, l'usine est déshumanisante, puisqu'on demande aux humains de renoncer à leur dignité et à se soumettre à la machine, qui devient la mesure de toutes choses (opposé à "L'homme est à la mesure de toute chose" humaniste), face au balancier, elle dit "J'ai compris pour quoi je comptais, zéro"

\subsubsection{Le capitalisme et la démesure}
Simone Weil souligne un paradoxe : l'univers capitaliste est obsédé par la mesure, tout en étant en quête d'un profit indéfini et infini, donc par la démesure. Cette démesure s'impose dans le fait qu'il n'y a pas de limite à l'intensité du travail et que l'entrepreneur cherche à aller toujours plus loin en la matière. Dans ce cadre, l'ouvrier n'est qu'une variable d'ajustement, une source d'énergie. C'est la négation de la pensée de Kant selon laquelle il faut toujours traiter autrui comme une fin et jamais comme un moyen.

\subsubsection{La mort comme limite}
Le capitalisme industriel en définitive ne connaît qu'une limite. C'est la mort. D'où l'usage du paradigme du coureur de Marathon, l'analogie entre l'intensité toujours plus élevée et la course. Revoir l'analogie possible avec le stakhanovisme. L'ouvrier peut travailler jusqu'à la mort. Cynique, Weil dit que l'entrepreneur ne se rendrait compte du problème que si la mortalité ouvrière devenait tellement forte que ça deviendrait difficile de recruter.

\section{Analyse d'extrait (418-421)}
Cet extrait est très différent des analyses du machinisme de Weil, plus philosophique.
\par Le début du texte est une affirmation qui conteste l'espérance révolutionnaire (les intellectuels socialistes qui ont vu dans la révolution la possibilité d'une société juste, ont repris le terme d'espérance, une des vertus théologales du christianisme, opposée au désespoir et au suicide ; pour ces intellectuels, cette espérance sera que la société permettra aux individus de "réaliser pleinement leur humanité", et alors le travail nécessaire ne durera pas toute la journée, et la tâche productive ne sera plus un identificateur). Le socialisme réel désigne ce qui s'est passé dans les pays qui se disaient socialistes (bloc de l'est, Chine de Zedong, Cuba...) et qui n'a rien à voir avec les idées de Marx, et Weil écrit à l'époque de Staline, affirmant que le bolchévisme n'a rien changé à la condition des ouvriers, puisque les dirigeants ont juste été remplacés. Parmi les lecteurs de Weil en 42, il reste encore beaucoup de gens pour croire en l'espérance révolutionnaire.
\par Mais Weil s'y oppose : pour elle le travail d'exécution sera condamné à la servitude, à l'esclavage. C'est un travail lié à la nécessité (ce à quoi on ne peut se soustraire, contraire de la liberté), opposé à la finalité (pour que l'homme ait une existence digne, il doit pouvoir se fixer des buts qui l'élèvent au-dessus de la vie matérielle). Weil considère qu'on va pouvoir sauver les ouvriers du malheur en leur permettant d'accéder à la "joie" (sens chrétien, perception de la proximité de Dieu et d'autrui qui donne une couleur permanente à l'existence) et à la "beauté" (Weil pourrait penser à la beauté de la nature, beauté de la Création et reflet de l'harmonie de Dieu). Weil n'envisage pas que le travailleur puisse échapper à cet enfermement dans le besoin. Par exemple, les éboueurs seront toujours nécessaires, sous la forme d'une caste d'intouchables (personnes qu'on ne peut pas toucher au risque de se souiller) comme en Inde ou d'un métier (en France, ils sont payés "convenablement" pour compenser le contact à la souillure et empêcher leur discrimination).
\par Parce qu'il faut bien vivre, l'ouvrier accepte un travail qui a une dimension déshumanisante, affirmée de deux manières. D'abord avec l'image de l'écureuil qui tourne dans sa cage, qui passe beaucoup de temps à faire tourner une roue parce qu'il manque d'activité. Weil déshumanise le travailleur, en le réduisant à un animal décérébré, enfermé dans une répétition stérile. C'est un prolongement du fait qu'elle considère impossible l'insertion du désir dans le travail ouvrier. Le désir réfère ici à la capacité à se projeté dans un système d'attentes, parce qu'il y a quelque chose au bout (c'est une idée platonicienne qui correspond à la notion de projet contemporaine). Weil affirme que l'ouvrier est enfermé dans le présent, à l'échelle de la journée, une séquence décomposée en répétitions des mêmes gestes. Elle explique la servitude comme une situation où il n'y a aucune possibilité d'évolution, le travail étant servitude puisqu'il ne permet que de maintenir la vie viologique. Gautier disait en 1936, après des critiques de l'économie"Mais s'empêcher de mourir, cela s'appelle-t-il vivre ?", rejoignant le propos de Weil. Le rapport au temps déshumanise de deux manières : dans un temps présent de répétitions, et un temps futur sans aucune espérance. 
\par Depuis la philosophie grecque antique, on considère que l'existence doit être guidée par la quête du Bien (rien à voir avec les biens de consommation). Un travail digne de ce nom selon Weil est celui auquel on peut associer le Bien, comme avec la contribution au fonctionnement de la collectivité. L'ouvrier ne peut pas recourir à cette démarche, puisqu'il ne voit pas la destination finale de ce qu'il fait.
\par Dans le deuxième et troisième paragraphe, elle utilise le vocabulaire de l'écoeurement et de la lassitude. Ce sont des états permanents, indissociables de la condition d'ouvrier. Il y a un écoeurement de l'existence et de soi-même : comment supporter cette existence sans perspective, sans rapport avec le bien ? Elle ajoute que la tentation de la mort est plus forte pour les ouvriers plus sensibles et intelligents. Ce qui définit la condition ouvrière est si dur que même s'il y a des variations selon les jours, ça ne change rien au ressenti.
\par Dans le paragraphe suivant, elle développe la question des biens. Quand la poursuite du Bien et des biens devient impossible, elle rappelle d'abord que l'existence ne peut pas être une fin en soi (il faudrait distinguer la vie au sens biologique et au sens d'avoir accès à la finalité), et explique que les biens doivent s'ajouter à la vie pour avoir une existence digne de ce nom. Sans aucun rapport au bien, à la finalité, alors la vie devient un mal en elle-même, l'unique objet du désir. Il faut distinguer dans le propos de l'existence la survie, la vie et l'existnece. Puisque l'âme est dans l'état de la survie, elle est dans l'horreur, devenant un mal.
\par Pour illustrer cette exemple, dans le paragraphe suivant, elle fait intervenir le paradigme de l'esclavage. Elle recourt à une espèce de résumé synthétique de l'esclavage. Sur le point de mourir, vous n'avez pas de travail. Mais on vous embauche, et vous devez travailler pour survivre. Le paradigme qu'évoque Weil est un homme fait esclave après avoir perdu un combat, et épuiser son énergie en efforts. La logique du recours au paradigme est de donner une illustration claire de l'horreur dont elle parle. Elle joue sur les acceptions antiques du mot âme et du sens chrétien, dans sa citation qui dit que les esclaves ont perdu la moitié de leur âme.
\par Elle fait ensuite le lien entre les situations des ouvriers et des esclaves. Elle dénonce le travail dans l'usine de manière radicale. Ensuite, elle donne un nouvel exemple pour illustrer la répétion déshumanisante : elle compare l'ouvrier à une balle, une mécanique qui prend la place de ce qui est en vie, qui doit être associé à un projet. Ele répète qu'on doit travailler pour manger. Elle n'évite pas la répétion, mais s'en sert pour insister sur le cercle déshumanisant du travail.
\par Dans le paragraphe suivant, elle parle de démoralisation. Weil prend le mot dans un sens plus fort que celui qu'on peut utiliser aujourd'hui : elle veut dire ne plus être en mesure de satisfaire des exigences morales. Simone de Beauvoir, dans la question de la moralité du peuple, raconte qu'on lui a parlé d'un peuple sans accès à la moralité, qui vivait sans dignité ni existence affective. On considérait le peuple comme bestial, démoralisé. Cette démoralisation est la conséquence inévitable de la condition ouvrière telle qu'elle est produite par l'usine. On pourrait se demander pourquoi en-dehors de la sphère de l'usine, le peuple n'est pas démoralisé, ce qui prépare ce que Simone Weil dira dans son dernier article, en disant comment ne pas couper les travailleurs de la joie.
\par Dans le dernier paragraphe, elle explique comment les ouvriers survivent. Elle explique que c'est grâce à des scénarios d'évasion illusoire, soit par une insensibilité ("inertie morale") ou par une force physique qui permettent de supporter le vide. Pour ceux qui n'ont pas cette force, il faut trouver des compensations pour survivre. La première compensation qu'elle évoque c'est le rêve d'une autre condition sociale pour soi ou ses enfants, sortir de la condition ouvrière sans l'améliorer. C'est une illusion, puisque peu de gens échappent à la reproduction sociale. Ensuite sont les plaisirs faciles et violents ("le rêve au lieu de l'ambition"), qui veut dire ne pas transformer son statut, un désir d'évasion qui repose sur des stratégies d'illusion, ce qui mène Weil à comparer cette méthode à l'usage de stupéfiants, qui altèrent la conscience pour que l'individu ait l'impression de s'échapper de la réalité. Elle prend l'exemple tangible des beaux habits du dimanche pour expliquer comment on satisfait sa vanité. Et ça veut dire se donner une impression de puissance, une licence (transgression des règles sociales et morales) par la débauche. Weil suggère que les activités sexuelles des ouvriers sont débordantes et désordonnées, parce que ça leur procure une forme d'évasion. Elle fait l'analogie avec les stupéfiants, en concédant que c'est lié à la souffrance des ouvriers.
\par Il y a un dernier recours, la révolution. Et pour elle, c'est encore une forme d'ambition, mais pour un collectif. Pour elle, ça ne peut pas marcher, et elle prend l'exemple de l'URSS, dont la révolution socialiste n'a pas changé la condition des ouvriers, ou en tous cas pas pour le meilleur.
\par Dans ce qui suit cet extrait, elle continue sur la révolution. Dans une lettre précédente, elle dit qu'elle n'est pas hostile à la révolution, si c'est un moyen d'améliorer la condition des ouvriers, mais elle nourrit des réticences à la faire, d'abord parce que la révolution est souvent juste une rêverie creuse, ensuite parce que comme le montre l'exemple du bolchévisme, la révolution ne change pas nécessairement la condition ouvrière et enfin parce que la révolution procède d'une exacerbation des tensions sociales (lutte des classes marxistes) et est donc un processus violent ("La révolution n'est pas un dîner de gala", litote de Zedong). Elle y préfère quelque chose qu'on pourrait appeler une collaboration.
\par Dans le premier paragraphe après l'extrait étudié, elle dit que le fait que les ouvriers aient encore une capacité de révolte est admirable, mais que peut se développer un impérialisme ouvrier, tout comme les impérialismes nationaux existent. Elle s'en prend à une certaine tendance du socialisme qui dit que la majorité ouvrière doit imposer ses idées. Pour elle, ça consiste à remplacer une domination par une autre. Pendant qu'elle écrit, elle voit plusieurs impérialismes européens s'affronter pour dominer le monde. Et surtout, elle considère ça absurde, puisque les ouvriers n'ont pas les "qualités" pour dominer. Elle considère que l'espérance révolutionnaire est en fait un opium du peuple, comme ce que Marx dénonçait dans la révolution. Elle conteste que ceux qui sont voués à des tâches d'exécution changent radicalement de status.
\par Dans le paragraphe suivant, elle évoque un côté exaltant de la révolution, une aventure collective qui sort de la nécessité, qui veut dire une ouverture à d'autres perspectives. C'est pour elle une des causes de l'attrait de l'idée de la révolution sur les ouvriers. Elle fait une référence aux romans policiers qui attirent les ouvriers, comme forme d'ouvertures d'un possible.
\par La suite est plus importante. Historiquement, Guisot, dans les années 1830 lance l'injonction "Enrichissez-vous par le travail et par l'épargne." Il veut que les ouvriers deviennent des bourgeois, ce qui leur donnerait le pouvoir de voter dans un régime avec un suffrage censitaire. Donc les bourgeois ont voulu que les ouvriers suivent leur volonté d'acquisition d'argent. Cela s'est fait avec le travail aux pièces, qui rendait une quantité proportionnelle aux efforts d'argent. Mais comme la condition est telle que personne ne peut s'enrichir, ça mène à une frustration intense, qui peut provoquer des explosions de violence, des révolutions. Donc la bourgeoisie a commis une erreur. Un petit commerçant, un petit industriel peuvent devenir très riches, mais restent dans la même sphère. Alors que les ouvriers ne peuvent pas changer de sphères, peu importe le nombre de pièces produites. Des autres sphères, comme les professeurs et les artistes ont d'autres intérêts que l'argent, et donc n'y prêtent pas d'intérêt, qu'ils soient riches ou pauvres. Alors qu'un ouvrier doit quitter la condition ouvrière s'il doit devenir riche. Il y a une impasse politico-sociale : la bourgeoisie propose aux ouvriers de faire comme eux, ce qui n'est pas possible, et ça provoque des tensions.
















\chapter{Tangentes}
Il faudrait lire les textes de Balzac comme des notices d'aspirateur.
\par Obélix est un bon exemple de propos peu pertinent en tant qu'ouverture.
\par Jean-Luc Godard est mort le 13 septembre 2022. Théoricien du cinéma et cinéaste de la Nouvelle Vague, un auteur très important. Il a sorti plusieurs films indispensables recommandés par le professeur : \textit{À Bout de Souffle}, \textit{Le Mépris} (seul grand rôle de Bardot), \textit{Pierrot le Fou}, \textit{Une Femme est une Femme} (comédie).
\par Sur l'actualité toujours, le professeur revient sur le "Le PCF ne veut pas être le parti de l'assistanat" du secrétaire général du PCF Fabien Roussel pendant la fête de l'huma (le journal nommé l'Humanité, de Jaurès au XXe). Il y a une querelle dans le débat français sur la figure fantasmée des assistés, des gens qui vivraient seulement des allocations (prestations versées par l'état à des gens dans une situation financière ou physique spécifique). La logique qui justifie les allocations c'est celle d'un Etat-Providence qui redistribue la richesse (les richesses sont reprises par impôts [revenu, succession] et sont réallouées). La redistribution n'est pas motivée par la seule philanthropie mais aussi par la peur des révoltes du XIXe et le fantasme du "Grand Soir", l'explosion de violence qui détruirait la société (imagée dans la révolte à la fin de Germinal). La république va donc mettre en place un régime de redistribution par le biais d'un Etat-Providence. Le problème de cette providence serait que les gens, plutôt que de rechercher un travail/emploi (cf conférence) et la dignité d'un statut de travailleur, préféreraient la paresse ("La France de ceux qui se lèvent tôt" de Sarkozy contre celle de ceux qui ne se lèvent pas). Il y a en effet des personnes qui n'ont pas eu d'emploi depuis longtemps et désocialisés, qui ne peuvent plus retrouver de travail (imagée dans ce dessin de Wolinsky dans les années 80 avec un enfant qui dit que plus tard, il veut être "chômeur comme papa", montrant une reproduction sociale). Il y a un débat de sociologues sur l'existence d'une culture de la pauvreté. Mais statistiquement, est-ce significatif ? Les sociologues seuls le savent. Il y a au même moment une crise du travail, avec les "working poors", des gens travaillent à temps partiel (statistiquement, plus de femmes en temps partiel) dans un emploi mal rémunéré, mais qui ne peuvent pas se permettre un logement. En effet, les logements sont très chers. Un idéal démocratique est qu'un salaire devrait garantir une dignité citoyenne, ce qui constitue un drame humain. Dans cette "culture de l'assistanat", on évoque souvent le RSA, accessible après 25 ans seulement, une allocation minimale. Cependant, plus de 30 pourcents des personnes éligibles ne le demandent pas. D'abord puisque les gens ont des problèmes pour faire des démarches, en étant illettrés ou sans connaissances numériques, et ensuite une forme de honte, celle d'être considéré comme un exclu, un chômeur. Ainsi, les gens parlant de la culture de l'assistanat sont souvent bien trop tranchés. Cela se calque sur les clivages politiques, la droite néolibérale s'y opposant le plus et les partis plus sociaux défendant ces "assistés". Mais Roussel constitue une exception. Une explication commune est que le PCF est très faible aujourd'hui, ses anciens électeurs votant notamment au RN, alors que le PCF s'annonce contre l'assistanat, c'est attirer des voies plus centristes que leur démographie d'origine.
\par Sur le Grand Soir : le XIXe siècle est marqué par l'espérance révolutionnaire. Beaucoup de penseurs classés comme socialistes envisagent des transformations radicales de la société. Le plus connu d'entre eux est Marx, qui élabore une nouvelle structure qui transformerait l'histoire de l'humanité et du travail aussi, débarrassé du profit. Dans sa société communiste, le travail sert à combler les besoins : "on sera cordonnier le matin et poète l'après-midi". Le travail n'absorbera ainsi plus toute la vie de l'individu. Le cordonnier poète en est un bon exemple, alliant un métier dévalorisé et un métier survalorisé. Mais cette transition n'est pas automatique, impliquant une phase de violence appelée la révolution du prolétariat (ensemble des personnes travaillant sans posséder). À cause de la baisse tendancielle du taux de profit, le capitalisme est forcé de ne faire qu'aliéner encore plus ses salariés. La révolution, menée par une avant-garde (ce qui lui donne un projet, contrairement à la révolte qui est simplement violente), construira une société égalitaire. Le Grand Soir c'est celui de la révolution, évocation inquiétante pour les bourgeois, que le défilé des mineurs de Germinal incarne à la perfection. Zola garde une grande ambiguïté, les mineurs représentant une forme d'humanité dégénérée.
\par L'assemblée a voulu passer une loi qui testerait l'alcoolémie des chasseurs. L'argumentation du chef de la fédération française des chasseurs fut de dire que les cyclistes et les gens dans l'administration française buvaient aussi. D'une, un cycliste qui boit a du mal à tenir, est soumis à des contrôles d'alcoolémie et surtout n'a pas une arme mortelle. Quant aux exemples de professeurs qui aient bu de l'alcool et tué un élève, ils semblent ne pas exister. Il n'y a pas de comparaisons à faire entre les dangers que pose un chasseur ivre et un cycliste ivre.
\par 28/09 Le prof a commencé par nous dire d'aller nous renseigner sur les dernières politiques de l'UE quant au moteur à hydrogène.
\par Ensuite, le prof nous a dit qu'il fallait aller se renseigner sur le module de la NASA avec les astéroïdes.
\par Ensuite, le prof a parlé d'un entretin avec une élève, et l'a forcé à lui expliquer le principe du ventilateur. Donc faut se préparer aux entretiens en s'intéressant au principe de fonctionnement de tous les trucs qu'on pourra mentionner en entretien.
\par 05/10 Le prof a commencé par la publication d'une tribune signée par 400 philosophes et éthologues, la déclaration de Montréal qui repense le rapport avec les animaux. C'est une version radicale en faveur des droits des des animaux. La déclaration est à courte et à lire. Lire aussi le manifeste animaliste de Corinne Pelluchon, livre bref sur les mêmes thématiques.
\par Le fameux EPR finlandais a coûté cher et a eu douze ans de délais, mais il est en fonctionnement. On peut aussi signaler des sites de production d'hydrogène.
\par Cela fait cinq ans que le mouvement Metoo a commencé, et c'est un sujet possible en oral.
\par 12/10 : le prof a commencé en mentionnant le prix Nobel de physique a été attribué à un français, Alain Aspect.
\par Il a mentionné les problèmes avec le pétrole et son prix. Dans l'actualité il y a quelque chose de très intéressant avec le pétrole, et c'est qu'il y a des grèves dans des raffineries de pétrole. Les grèves consistent à prendre en otage quelque chose, quan don est employés. Quand des professeurs le font, c'est peu important, quand des ouvriers de raffinerie le font, l'état est plus embêté. On commence à parler de réquisition, de la nécessité de laisser les entreprises fonctioner en assurant la circulation du pétrole, mais d'un autre côté, le droit de grève est constitutionnel.
\par Il a mentionné ce qui se passe en Iran, une situation marquée par de grandes tensions. Une jeune femme a été arrêtée par la police des moeurs pour avoir mal porté son voile puis torturée en prison, ce qui a causé des grandes manifestations. La question est de savoir comment la répression et la quasi-insurrection vont continuer. On se demande s'il y aura des changements de régimes. Il faut aussi se renseigner sur l'histoire des changements d'état en Iran.
\par Il a recommandé le fait de lire la revue "L'Usine Nouvelle". Pour y accéder, voir les documentalisters et demander l'accès Europresse.
\par 09/11 : Il a commencé par nous demander de nous intéresser à la COP27 qui a lieu en ce moment en Egypte
\par Il nous a aussi invité à nous intéresser à la chronique de France Q l'indice de réparabilité par pour combattre l'obsolescence programmée. On impose aussi au fabriquant un indice de durabilité, qui d
\par Le PDG de Total perçoit un salaire annuel de 6 millions d'euros. Comme il pratique l'optimisation fiscale, il donne bien moins de la moitié de son salaire à l'état (ce qu'il lui devrait). On peut donc se demander ce qu'il fait de son argent (après tout, "On ne peut manger du caviar que deux fois par jour.") L'industrie du luxe, onéreuse par le temps de travail est les matériaux nécessaire pour construire leurs produits, est nécessaire aux grands salaires. Pour ne pas perdre la face, un concept introduit par Bourdieu, la distinction, dit qu'on dépenserait alors beaucoup dans le luxe pour garder une apparence riche. Dans la théorie du ruissellement, c'est une bonne situation : en achetant une voiture très cher, alors ça veut dire qu'il a fallu payer beaucoup de gens pour la construire, et donc pour le luxe. Dans la théorie économique du gâteau, on a que le salaire immense des PDG a été pris quelque part d'autre. George Marchais a dit dans les années 75 : "Le problème de Mr [Machin] c'est de ne pas savoir comment faire pour dépenser moins d'un SMIC par jour". On pose la question du salaire. Tout travail mérite salaire, et si le salaire d'un ingénieur et d'un balayeur sont similaires, dans une société capitaliste il n'y a pas d'intérêt à faire ingénieur. Aux yeux du capitalisme, les professeurs du collège de France (fondé par François 1er pour s'opposer à la Sorbonne catholique) sont moins intéressants que Cyril Hanouna. Qu'est-ce qu'un salaire ? Il est bon que chacun ait un salaire pour mener une vie conforme aux exigences de la dignité humaine et si possible de satisfaire des exigences de confort humain. Un salaire est destiné à couvrir les besoins fondamentaux. Dans une société parfaite, c'est ce que les salaires permettraient de faire (dans la nôtre, il y aurait un problème avec les SDF et les restos du coeur). Une légende dit que l'écart des salaires en Suède dans les années 80 était de 7, bien plus petit que le nôtre. Mais c'est impossible dans les sociétés libérales : les actionnaires ont élu le PDG de Total pour gagner des dividendes, et ils doivent le payer cher s'ils veulent le garder. Dans une société mondialisée, interdire les paradis fiscaux dans un seul pays n'est pas suffisant. On ne peut pas régler le problème des salaires, parce qu'il y a un marché. C'est ce qui mène aux 80 millions d'euros par an pour les footballeurs les mieux payés. La question des salaires ne peut être appréhendée dans notre société actuelle. Et ça nous mène à l'idée en apparence scandaleuse d'un homme qui a un salaire immense, que ses qualités ne sauraient justifier.
\par Le couple mesure/démesure est fondamental dans la philosophie depuis la grèce antique. L'humanisme, défini par Rabelais, dit que "L'homme est la mesure de toute chose." Cette idée s'oppose aux fanatiques qui disent que Dieu est la mesure de toutes choses. Cela mène au terrorisme religieux, quand on décide qu'on est la personne qui a compris ce que disait Dieu. C'est une démesure du point de vue humaniste. Dans la situation de salaires extravagants, il y a de la démesure au point de se demander quelle est l'utilité de ces salaires (on en vient à payer des produits de luxe sans utilité propre, seulement pour établir sa richesse (escalier en platine)). Quand on dit qu'un salaire doit servir à vivre dignement, on établit une mesure, et le salaire de six millions par ans est une forme de démesure.
\par 16/11 : On dit toujours que l'européen est incorruptible. Mais non, il est cher !
\par 23/11 : dans l'émission \textit{Les pieds sur Terre} (quelqu'un qui raconte une partie de sa vie) sur France Culture, pendant l'intervention de vendredi, on a eu la "confession" d'un DRH qui a passé sa vie à virer des gens, jusqu'à devoir virer une dame qui n'aurait pas pu trouver autre chose et qui n'avait rien fait de mal. Donc il a été viré. Dans sa vie, il a d'abord été soldat dans l'armée belge, avant de démissionner, puis de devenir ingénieur et finir par atteindre un poste de DRH (comme si les humains étaient une ressource). Il parle des rapports de forces, de comment on peut faire pour que les gens tentent de s'en sortir le moins mal possible, mais de comment ça reste difficile. Cette intervention aurait lancé une polémique que le professeur n'a pas suivi. Le livre de Florence Aubenas, \textit{Le Quai de Ouistréam}, est du journalisme immersif qui permet de comprendre le monde du travail d'un point de vue très différent.
\par 30/11 : il a recommandé l'émission LSD, qui a consacré une semaine à Toutankhâmon. Il a surtout recommandé la dernière émission, qui parlait de comment la politique égyptienne se sert de ce pharaon comme symbole. En effet, il y a un phénomène de réappropriation nationaliste de l'histoire des pharaons.
\par Il recommande aussi dans "Avec Philosophie", la série d'émissions "Peut-on penser la race sans l'essentialiser". La première et la deuxième émissions étaient les plus importantes, et on écoute dedans le discours de Nicolas Sarkozy à Dakkar (écrit par Claude Guéant), consternant d'ignorance et de pensée néo-coloniale. (Un collectif d'historiens y avaient répondu avec "L'histoire africaine expliquée à Nicolas Sarkozy")
\par Il recommanda aussi "Le pourquoi du comment", une courte chronique (4 minutes) qui intervient à la fin d'autres émissions, dans le même thème que l'émission. Il recommande le pourquoi du comment après l'émission "Le cours de l'histoire" et "Avec Philosophie" (tenue par un professeur d'Ulm).
\par 07/12 : il a conseillé de parler de la pénurie d'électricité. Dans un oral, on pourrait nous demander pourquoi un pays comme la France risquent des délestages, là où c'est plutôt associé à des pays "en voie de développement". On parle aussi d'une sixième extinction de masse. Il faudrait lire les dernières études sur la pauvreté : il y aurait en France 1.5 millions de personnes qui vivent en-dessous du seuil de pauvreté, la moitié d'entre elles ayant moins de 25 ans (le RSA ne commence qu'à 25 ans). Les banques alimentaires viennent de faire leurs collectes.
\par Dans un quartier de Berlin avec des rues qui avaient des noms de colons allemands, on a remplacé les noms de rues par des noms de victimes du colonialisme. Les associations des diasporas africaines présentes à Berlin affirment qu'il s'agit d'un dédommagement symbolique et d'une opportunité éducative. On peut se demander si on devrait garder la statue de Bugeot au Louvres, après ses crimes de guerre en Algérie. Le "guide du Marseille colonial" est en train de sortir pour parler du rôle de Marseille dans la colonisation française.


\section{Sur le travail}
\par Notre temps est structurée entre le temps de travail et le temps libre. Le covid a cassé les distinctions entre les deux pour les personnes normales, mais dans la prépa les deux sont déjà confondus. La nécessité d'un endroit où travailler calmement est une évidence. Cette question de la frontière peut être élargie : un bulletin d'information écouté sur lequel on va dire quelque chose en cours doit-il être compté dans le travail ? Est-ce qu'en se promenant et en repensant à son résumé de dissertation, doit-on insister pour être payé ? Cependant, ce n'est la situation que de la prépa. Par exemple, les partis de gauche qui ont protesté contre les ouvertures des magasins le dimanche, ont raison en affirmant que c'est mauvais pour les gens et leurs vies de famille, quand bien même ça serait payé double. Doit-on cautionner une sitaution où les gens soient obligés de travailler le dimanche, pour voir encore moins sa famille et ses amis ?
\par Qu'est-ce qui est du travail, et qu'est-ce qui n'en est pas ? Est-ce que l'esprit sain dans un corps sain mène à des dérives financières comme le salaires absurdes des professionnels du sport, la tricherie qui peut s'organiser, les dopages organisés pour maintenir la société du spectacle. Comparé aux sportifs dans les sociétés du bloc de l'est, les sportifs de haut niveau étaient des personnages militaires, des égéries qu'on faisait souffrir pendant des expériences et des dopages pour maximiser leurs performances. Et quid des travailleurs du sexe ? La prostitution est-elle un travail ? On en parle comme le plus vieux métier du monde, puisqu'il y en a dans la majorité des sociétés, et ainsi il faut les régler sans être idéalistes, "Qui veut faire l'ange fait la bête" pour citer Pascal. La nature de ces gestes est particulière, et la décrire comme un travail est source de questions. Et à propos des mères porteuses ? Est-ce qu'on devrait être rémunéré pour porter les enfants d'autrui ? Est-ce que le corps est marchandisé ? Un débat de principe et aussi de réalisme doit se faire, car si ça devient suffisamment sûr, ça sera nécessairement fait alors autant légiférer et rendre la chose légale. "Réaliste, la loi a horreur du désordre." Dans la loi de Veil (l'autre Simone), il est mentionné qu'il y a suffisamment d'avortements que le légaliser est le plus simple. Devrait-on aussi considérer les tâches ménagères et familiales comme une forme de travail ? Devrait-on rémunérer cette fonction, mais alors ça veut dire insérer l'argent dans la relation de l'amour filial.

\section{Troubles dans le travail}
Notre président avait dit dans sa campagne que notre société serait régie par la valeur travail. Aujourd'hui, d'après un journaliste, on considère que le poids de la valeur travail est tel qu'on ne peut plus dire qu'un livre est nul puisqu'il a été produit par un travail. Le poids de la valeur travail est donc contre-productif : valoriser la quantité de travail mène à dévaloriser la qualité. Nos sociétés "démocratiques" et humanistes sont attachées à la valeur travail,  mais il faut être prudent face à elle, qui finirait par avoir des conséquences néfastes. 


\end{document}