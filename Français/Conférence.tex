\documentclass[a4paper,12pt]{book}
\usepackage{ae}
\usepackage{aeguill}
\usepackage{amsthm}
\usepackage[utf8]{inputenc}
\usepackage[T1]{fontenc}
\usepackage[french]{babel}
\usepackage[utf8]{inputenc}
\usepackage{graphicx}
\usepackage{hyperref}
\usepackage{xcolor}
\usepackage[left=1cm, right= 1cm, top=2cm, bottom = 2cm]{geometry}
\usepackage{array,multirow}
\usepackage{amsmath,amsthm,amssymb}
\usepackage{amsfonts}
\usepackage{stmaryrd}
\usepackage{tcolorbox}
\usepackage{lmodern}



\renewcommand{\thechapter}{\Roman{chapter}}
\renewcommand{\thesection}{\Roman{section}}
\renewcommand{\thesubsection}{\Roman{section}.\arabic{subsection}}
\renewcommand{\thesubsubsection}{\Roman{section}.\arabic{subsection}.\Alph{subsubsection}}

\title{Conférence travail 8 septembre}
\author{Mehdi Arrignon}
\begin{document}
\section{Introduction}
Citation : "'Richesse, c'est pouvoir" a dit Hobbes, et le genre de pouvoir que cette possession transmet directement, c'est le pouvoir d'acheter. C'est un droit decommandementsur tout le travail d'autrui." Adam Smith.
\par Adam Smith est un père de l'économie politique classique, philosophe et économiste écossais, ayant écrit "La théorie des sentiments moraux" et "Réflexions sur la nature et la cause de la richesse des nations". Il a passé dix années de sa vie à écrire cette dernière, et on lit une traduction de Lucak, édition québécoise trouvable en ligne.
\par Adam Smith n'est pas le premier à réfléchir sur le travail et la richesse, par exemple les auteurs mercantilistes comme Jean Bodin, où le ministre Colbert considéraient que la richesse venait de la possession de ce qui est rare. Leur doctrine, du XVIe au XVIIe, pensait que la richesse venait de l'afflux d'or et d'argent dans la caisse de l'état, doctrine qui a causé des crises pour la cour espagnole, qui voulait absolument avoir le plus d'argent.
\par Les physiocrates sont venus aussi, comme le Marquis de Miraguaux, Turgot. Ils représentent l'économie et le travail comme un circuit économique, et s'accordent sur le libéralisme. Un d'entre eux définit trois classes de travailleurs : productive (agriculteurs), stérile (artisans, marchands) et propriétaires. Les agriculteurs produisent de la valeur, qui est ensuite répartie entre les stériles et les propriétaires. En période de diète, les physiocrates argumentent que c'est le blé qui compte, pas l'or. On trouve des traces de cette pensée dans l'économie moderne, avec la division entre un premier secteur d'extraction de ressources, un deuxième de production d'objets et un troisième de services.
\par Qu'apporte Adam Smith ? Il apporte la valeur travail. Pour lui, ce n'est pas l'or ou le blé qui est source de richesse mais le travail. Il sous-titre son livre "des causes qui ont façonné les facultés productives du travail et de l'ordre suivant lequel ses produits se distribuent naturellement dans les différentes classes du peuple". Tout ce qu'il écrit, c'est des promesses d'un monde meilleur avec un travail différent. Pour lui, les gains se distribueront naturellement dans les classes. Le travail c'est donc le bonheur pour lui. Ayant quitté les vacances et les sandales pour se permettre un zeugma.
\par Cette conférence va tenter de mettre un point de vue différent de ceux des professeurs. Elle traitera de 1776 jusqu'à nos jours, pour parler des penseurs qui ont conceptualisé le travail sur cette période. Il y a trois parties : d'abord la valeur travail dans l'économie (1776-1874), puis les apports de la sociologie (depuis fin XIXe) et enfin la valeur travail dans la science politique actuelle.

\section{Valeur travail : des économistes classiques aux néo-classiques}
On appelle économie classique l'ensemble des théories développées autour de trois grands auteurs, souvent en désaccord mais partageant le concept de la valeur travail. Ce sont Karl Marx, David Ricardo et Adam Smith.
\par Dans le premier texte d'Adam Smith, il annonce rapidement ce qu'il entend, en introduisant la division du travail et sa valeur : accroître la puissance politique et la production. La division du travail augmente l'efficacité, et a d'autres bienfaits, qui seront couverts dans son livre. Smith fait le parallèle entre le degré de développement d'une société et son degré de division du travail.
\par Adam Smith prend un exemple qu'il a fait connaître, celui de la manufacture d'épingles. Il commence par prendre l'exemple d'un ouvrier solitaire, qui pourrait au plus faire 20 épingles dans l'heure. Ensuite, il décrit les différents métiers dans lesquels sont divisés le processus de production. Smith montre qu'on peut diviser en dix-huit tâches la production d'une épingle. Il faudra donc plus d'ouvriers, mais alors dix d'entre eux produiraient 48000 épingles par jour, au lieu des 200 s'ils produisaient seuls. La division du travail permet d'augmenter de manière non-proportionnelle la productivité. Pour lui, la division permet d'améliorer grandement l'organisation.
\par Smith justifie ça de trois façons : d'abord, les tâches plus simples sont effectuées plus vite et avec plus d'habileté, ensuite on économise du temps en ne changeant pas de tâche (réduction de la flânerie) et enfin parce que la spécialisation permet plus aisément la mécanisation. Une tâche entière n'est pas facile à mécaniser, mais en la découpant, on peut construire des machines qui contribuent chacune un petit peu.
\par La question des limites de la division du travail : est-elle applicable partout ? Smith l'envisage, et finit par dire que seule l'industrie peut voir son travail divisé. En effet, on ne pourrait pas diviser le travail de la terre, puisqu'il y a un rapport saisonnier plus complexe. Pour les services, il considère là encore qu'il n'y a pas de division du travail possible. Aujourd'hui, le degré de technicité de l'agronomie demande une division du travail ; Smith aurait pu étendre sa notion à l'agriculture. En termes de services, on peut aussi faire des divisions du travail (call center).
\par La question des limites de la division du travail : existe-t-il des effets vicieux ? Smith l'évoque mais considère la monotonie du travail comme non-aliénante ; l'ouvrier deviendrait plus intelligent à force d'utiliser sa machine, et serait capable d'inventer son remplacement mécanique. Smith raconte une anecdote d'un enfant qui avait le travail très simple d'ouvrir et de fermer une vanne qui apprit à faire un travail de cordelettes pour être libre. Cependant, Smith n'a pas prévu le chômage et le fait que le travail des enfants ait beaucoup de problèmes.
\par La question du sous-titre positif et du remplacement : À partir de la première crise de 1873 et de 1929, le chômage apparut. Pour lui, remplacer les gens par des machines a si peu de coût et permet une telle hausse de revenu qu'elle va forcément se distribuer à d'autres entreprises. Ainsi, un ouvrier ayant perdu son travail en retrouvera parce que le profit créera des opportunités d'embauches nouvelles. Adam Smith ne conçoit pas le chômage.
\par Dans le deuxième texte d'Adam Smith, il traite de la richesse, de ce que les économistes appelleront la valeur travail. Ce terme est source de beaucoup de contresens. Smith se place dans la situation théorique où un homme pourrait se procurer les biens qui lui sont nécessaires en les fabriquant. Mais à son époque déjà, il n'y a qu'une toute petite partie des biens dont on a besoin qu'on produit directement. On attend ces produits nécessaires du travail des autres. Dès lors qu'une économie est marchande, on échange son travail contre celui des autres (la quantité de pain que je ne mange pas me permet d'acheter les choses des autres).
\par "Le travail est la mesure de la valeur de toute marchandise." La vente de ses produits permet d'obtenir du revenu, et donc le temps de travail des autres. Cet argent épargnent la fatigue, et contiennent une valeur d'une certaine quantité de travail. Le travail est le prix de toute chose. Une possession d'argent n'est cependant pas un pouvoir politique. C'est simplement le pouvoir d'acheter, de commander le travail d'autrui.
\par Smith va donner naissance à la valeur travail : ce qui donne valeur à un objet, c'est la quantité de travail nécessaire pour le produire. Une table prenant deux fois plus de temps à produire qu'une chaise, elle vaudra deux fois plus. Le travail est le seul moyen de comparer les choses.
\par La question des limites du temps de travail pur : du blé demande plus de temps pour être produit qu'une table. Ce que Smith dit, c'est que certaines productions demandent moins de temps de travail par la mécanisation, mais comme il a fallu produire les machines, leur temps de production compte aussi avec des modifications.
\par Sur ces questions, Ricardo et même Marx vont être assez d'accord sur la définition. Les limites de la valeur travail sont venues avec Ricardo, son principal ouvrage "Principes de l'économie politique" en 1817, est plus pessimiste que celui de Smith. S'il pense que toute richesse vient du temps de travail, Smith pensait que la prospérité n'allait que s'aggrandir. Mais Ricardo voit l'économie de manière plus stationnaire, par le biais de l'influence de Malthus (son ami), il s'inquiète alors que l'augmentation des capacités productives ne suivent pas l'augmentation de la population (loi des rendements décroissants).
\par Ricardo en tire un certain pessimisme : la population augmentante, il faudra cultiver plus de terres, donner plus aux rentiers, et alors les ouvriers vont avoir besoin d'agmentations de salaires. Ricardo voit dans la "baisse tendancielle du taux de profit" une fatalité. La machine allait ralentir jusqu'à stagner. La théorie optimiste de Smith s'est finie. Ricardo et Smith voient tous deux la croissance comme positive et à chercher. La notion de profit tend à être éliminée par Ricardo, mais c'est juste parce qu'il considère ça comme une fatalité, rien de marxiste en lui. La solution de Ricardo est le commerce international.
\par Karl Marx arrive. Il est connu comme sociologue, historien et économiste, publiant les premiers volumes du Capital en 1877 et Engels publie les dernier après sa mort. L'aliénation du travail - idée reprise par Weil ou Chaplin - est l'idée selon laquelle le travailleur n'est pas épanouissant pour les travailleurs, étant dépossédé de la capacité de décider de quoi que ce soit sur le produit, il est asservi. C'est la critique la plus connue de Marx du capitalisme.
\par Cependant, Marx prend aussi au piège les libéraux précédents. Il est en effet d'accord sur la valeur travail, mais considère que si toute valeur vient du travail, demande pourquoi toute la richesse ne revient pas immédiatement au travailleurs. Pour lui, la rente est du vol. Le surtravail, concept qu'il développe, est qu'une partie des heures des employés ne seront pas payées, mais serviront à dégager un profit pour l'employeur. Pour lui, c'est de l'extorsion de valeur et de temps de travail. C'est la base de la critique marxiste de l'idée de profit.
\par Il y préfère un modèle coopératif où les moyens de productions appartiennent à ceux qui travaillent. Il met l'accent sur la contradiction entre les bourgeois qui possèdent et ne travaillent pas et entre les ouvriers qui travaillent mais ne possèdent pas. Les économistes d'après feront face à un problème : les idées de Marx découlent directement de la valeur travail, il faut l'abandonner.
\par C'est comme ça que naît la théorie économique néoclassique, développée notamment par Léon Walras. Pour lui, ce qui donnera de la valeur ne sera pas le temps de travail (le temps de travail d'un ingénieur et d'un ouvrier non-qualifié a-t-il vraiment la même valeur ?), mais la rencontre de l'offre et de la demande après 1874. Peu importe la qualité du produit et l'effort nécessaire, la valeur ne dépendra que de l'offre et de la demande. Plus la demande augmente, plus les producteurs voudront accroître l'offre. Le prix qui monte ordonne de produire plus. Le consommateur a le point de vue opposé : lorsque le prix baisse, il faut en acheter plus.
\par Le prix d'équilibre est la valeur d'un produit, à l'intersection des courbes de l'offre et de la demande. Cette valeur varie en fonction de la quantité de demandeurs et des offrants. La "révolution néoclassique", c'est l'abandon de la valeur, considérée comme subjective. "Je dis que les choses sont utiles dès qu'elles peuvent répondre à un besoin quelconque et en permettre la satisfaction. (...) qu'une substance soit recherchée par un médecin pour guérir une maladie ou par un assassin pour empoisonner sa famille, c'est une question très importante à d'autres points de vue, mais tout à fait indifférente au nôtre." On abandone complètement l'objectivité de la valeur.
\par Après Walras, l'économie s'émancipe de la réflexion morale et philosophique. Chez Marx et Smith il y avait des tons très moraux, puis les économistes vont décider d'abandonner ça.

\section{La sociologie du travail}
La sociologie enseigne des éléments descriptifs sur le travail.
\par Une première chose à retenir sur la France est qu'on considère que sur l'évolution du travail de la France, on voit une hausse du niveau général de qualification. L'augmentation du niveau moyen des diplômes augmente de plus en plus. On est passés de huit pourcents et demi de gens avec un diplôme supérieur jusqu'à cinquante-et-un pourcents.
\par On remarque aussi une tertiarisation : on a une transition d'une économie agricole, jusqu'à une évolution vers une économie industrielle puis de services après la seconde guerre mondiale. Après 1952, le tertiaire prend une part de plus en plus importante : soixante-seize pourcents des travailleurs sont aujourd'hui dans les services, tandis que l'agriculture devient trois pourcents, le secteur industriel résistant un peu plus.
\par Pour les professions et catégories socio-professionnelles, on a depuis le début du XXe une grande baisse des exploitants, une a que les patrons se réduisent légèrement. Les ouvriers restent presque constants, puis les employés et cadres ont commencé à croître, et sont désormais aussi nombreux que les ouvriers. (L'artisanat est compris dans d'autres catégories comme patrons et commerce.)
\par On voit aussi une féminisation très forte de l'emploi depuis 1962 : à l'époque, un employé sur trois était une femme, tandis que l'égalité est presque atteinte aujourd'hui.
\par La sociologie ne se concentre pas que sur les statistiques, mais aussi sur l'évolution de l'organisation du travail : "de la déshumanisation taylorienne à la surhumanisation managériale"(Danielle Lionhart, sociologue du travail). Le management industriel naît avec Taylor (continué avec Ford, etc), ingénieur consultant qui va essayer de réduire la "flânerie systématique" en chronométrant les tâches, en inventant une organisation où tout est mathématisé. Avec la Ford T, Henri Ford reprend ça et organise de manière encore plus extrême, avec une chaîne de montage chronométrée. Cf les temps modernes de Chaplin.
\par Cette logique se généralise, mais ne sera pas remise en cause avant 68 par les organisations de salariés. En effet, pendant les trente glorieuses, les syndicats ne remettront pas le contenu du travail en cause, et s'occuperont plus de chercher des primes et des augmentations. Les combats syndicaux alimentaient le modèle taylorien : les augmentations salariales permettaient de maximiser la consommation et de renforcer le taylorisme. Il y avait un \underline{consensus consumériste}. Mais la malédiction du travail finit par atteindre un point de non-retour, où on ne pouvait plus rien accepté. Lionhart cite Bourdieu et sa double vérité du travail : "il n'est pas rare que le travail procure en lui-même un profit..."
\par La sociologue dit que même si les salariés ressentent leur oppression, ils ressentent le besoin de s'y investir pour rendre leur vie meilleure après. Un ouvrier professionnel de trente neuf ans lui a dit "on est dirigés par des types qui n'ont plus de connaissances que nous". Il faut s'investir dans le travail pour combattre sa violence. Le travail est considéré par beaucoup comme le fondament de la vie sociale tout en abrutissant. Cette ambivalence de dénoncer le travail et de l'apprécier arriverait à un paroxysme en 68.
\par La révolte ouvrière est alors un rejet du consensus de perdre sa vie à la gagner. Il fallait son mot à dire sur la dignité, sur un sens dans un travail contraint. Le management contemporain, post-taylorien, va récupérer cette critique, qui voulait réhumaniser, re subjectiver le travail. Le patronat va renforcer la subordination tout en augmentant la socialisation. Elton Layo a fait des recherches sur le Taylorisme avec des expériencese et des centaines de milliers d'entretiens, et a montré que s'intéresser aux ouvriers, de les écouter, permettait d'augmenter leur efficacité. Et alors il y a eu une reconstruction post-taylorienne du managériat, en humanisant le travail, cosmétiquement ou dans les slogans.
\par Elle donne comme exemple de ça : le PDG du slogan d'Orange "rendre le salarié unique et le digital humain". Il y a là-dedans une révolution managériale, qui humanise autant que possible et fait attention au travailleur, qui a des locaux de meilleure qualité (salles de jv dans la Silicon Valley), qui maîtrise de son emploi du temps. Tout ça permet alors de limiter les conflits d'intérêt, et les directions font autant que possible pour se rapprocher de ses employés (DRH rebaptisé Direction de Richesse Humaine ou Direction qui Rend Heureux).
\par La modernité de ce taylorisme presque émotionnel, est encore dans le contrôle. Il faut contrôler les salariés, et cette autonomie se réduit à rendre opérationnelles des méthodes et des rentabilités conçues loin d'eux. "Dissimulé sous une image humaine (...) l'esprit Taylorien veille encore sur leur travail."

\section{Travail et emploi en science politique}
Les enquêtes sur les opinions permettent de renseigner sur comment les gens donnent du sens au travail. Lors de la campagne présidentielle, le candidat Macron avait dit "Aujourd'hui comme hier, le travail continue d'être notre boussole." Avant lui, Sarkozy avait fait une campagne sur la valeur travail. Hollande avait justifié de ne pas se présenter en disant qu'il n'avait pas réussi à inverser la courbe du chômage.
\par Les politiques de l'emploi ont évolué de manière à ce que tous les politiques considèrent le travail comme quelque chose de bien. Le travail n'est pas une valeur en voie de disparition. Presque 90 pourcents des européens considéraient en 99 que le travail était important dans leur vie. En 2008, c'était même un peu plus haut. 94 pourcents des français le disaient par exemple. Une enquête fut faite en 2018 mais les publications ne sont pas sorties.
\par Qu'est-ce qu'on retrouve au travail ? le plus souvent, c'est l'opportunité de faire quelque chose.

\section{Conclusion}
Le travail s'est transformé. On a perdu une conception naïve de la valeur, celle que plus quelque chose a été travaillé, plus elle est importante. Plus tard, on a perdu nos certitudes sur ce que constituait le travail efficace, le travail tayloriste n'étant pas aussi efficace que le management plus moderne. Pour finir, avec la science politique on a vu que le travail fatiguait et broyait, mais pouvait aussi rendre fier.









\end{document}